\chapter{Introduction\label{cha:introduction}}

\section{Formula SAE}

Formula SAE\nomenclature{SAE}{Society of Automotive Engineers} is an engineering student design competition organized by the Society of Automotive Engineers dating back to 1978 \cite{fsaehistory}. Students from the University of Manitoba have participated in the competition almost every year since 1985. The competition consists of designing and constructing a small, open-wheeled, formula-style race car.

The Formula SAE vehicle is a performance car built with the primary goal of doing well in the dynamic events at the yearly competitions. These events test the vehicles' abilities in acceleration, braking, and handling. 

\section{Motivation}

Many of the issues that directly affect the teams' performance at competition relate to driver training, feedback, and the tuneability of the car. Most of the mechanical systems on the car must currently be imprecisely hand-tuned, and are packaged in hard to reach places, and require body panels or the seat to be removed for access.

Our overall goal is thus to improve the precision, adjustability, and repeatability of adjustment, of various important mechanical systems in the car, and to improve the efficiency of testing. Highly desirable is a shorter driver-feedback-tuning loop in order to eliminate overshoot and undershoot in tuning, and to avoid other external disturbances.

A second major goal is to improve upon the transmission control systems of previous years' designs.

\section{Problem Definition}

The goal of this work is to tackle several distinct issues on the 2010 Formula SAE Vehicle:

\begin{itemize}
 \item electronic control of the transmission is to be improved over previous years' control;
 \item engine output power will be increased under specific conditions by dynamically varying the intake geometery;
 \item the front-to-back brake bias will be electronically adjustable;
 \item data from on-board systems will be accessible wirelessly;
 \item an easy to use yet powerful user interface will allow the driver to control on-board systems and provide feedback of critical vehicle parameters.
\end{itemize}

Improving electronic control of the transmission should result in decreased gear shift times, improved mechanical reliability of the transmission, decreased driver effort, and the elimination of any manual mechanical interaction with the transmission by the driver.

Successful introduction of variable intake geometery will widen the peak torque output from the engine, improving acceleration.

Electronically adjustable brake bias will improve tuneability by allowing the brake bias to be adjusted on the fly, even while the vehicle is in motion. It should reduce the need for manual adjustment and calibration.

Wireless telemetry data will improve vehicle performance and reliability by relaying critical vehicle operation data from various sensors to the pit crew, who can then make decisions regarding the various vechile parameters. The module should be able to broadcast to the pit area while the vehicle is competing in all of the dynamic events. It must operate with a minimum of data loss despite the distance and motion of the vehicle relative to the pit crew.

The driver interface should reduce driver effort and improve vehicle performance by allowing the driver to manage all the tuneable vehicle parameters from a simple interface located on the steering wheel, and by providing the driver with real-time feedback from the various vehicle sensors in an easy-to-read format.

\section{Strategy}

To meet the requirements of the project, a networked 4-module architecture was chosen. The architecture will be described in further detail in Chapter \ref{ch:design}, however to aide the reader in understanding the scope of the project, the modules will be introduced here. These 4 modules are:

\begin{itemize}
\item the transmission and engine control module;
\item the brake bias adjustment module;
\item the wireless telemetry module; and
\item the driver interface module.
\end{itemize}

Additionally, to facilitate testing of the network, a CAN testing module will be developed, consisting of an off-the-shelf AT90CAN128 development board, and custom software.

The design and implementation of each module require several steps:

\begin{itemize}

\item Absolute requirements and specifications for system parameters are first estabilshed. The electromechanical interface linking the control systems and mechanical systems are decided between our group and the mechanical engineers responsible for each respective system. The measurement requirements from the mechanical systems are established and the appropriate sensors are chosen.

\item Appropriate components for the electronic modules are chosen, and the circuits are designed. The schematic of each module is designed and an appropriate printed circuit board is laid out. The boards are manufactured and populated with components. Each board must be tested to ensure all wiring is correct and the functionality of all the components is correct.

\item Firmware for each module must be implemented once the hardware design is finished. Software libraries for various components are written and tested. The libraries are combined with a high-level control algorithm for each module.
 
\item Each module is tested in isolation to ensure proper functionality. 

\item The modules are then interconnected and tested again.

\end{itemize}

\section{Outline of Thesis}

This thesis is organized into 8 chapters:

After this introductory chapter, Ch. \ref{cha:background} provides background information describing the vehicle systems we will interact with, the problems associated with these systems we are attempting to solve, and a description of previous efforts to solve these problems. 

In Ch. \ref{cha:goals_and_requirements} the goals and requirements of the design work will be specified for each of the areas described in Ch. \ref{cha:background}. 

Chapter \ref{cha:design} describes our design to meet these requirements. 

Chapter \ref{cha:implementation} describes our implementation of the design.
