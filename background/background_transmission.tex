\section{Transmission}

\subsection{Overview}

The Formula SAE 2010 vehicle uses a 6-speed manual sequential transmission to transmit power from the engine to the drive train. As in other types of manual transmissions, the sequential transmission works by engaging and disengaging several sets of gears with shift forks to obtain different gear ratios. A cast steel drum inside the transmission, called the \emph{shift drum}, has a series of complex grooves machined into it, in which the shift forks ride. Rotating the drum causes the forks to engage one gear and disengage the other. A ratcheting mechanism allows a single external lever, called the \emph{shift lever}, to rotate the drum forward and back by a discrete amount. Each full movement of the lever forwards or backwards causes a gear change up or down, respectively \cite{HowtoManualTransmission, cbr600}.

\subsection{Clutch}

The clutch is actuated by a lever attached to the engine body. A pre-tensioned spring keeps the clutch engaged when there is no force on the lever. As force is applied to the lever, the clutch is gradually disengaged. The relationship between lever position the distance between the clutch plate and flywheel is non-linear. 

\subsection{Gear Selector}

The gear selector is actuated by a lever attached to the engine body. The gear selector may be rotated in either direction from its rest point. Up-shifting is accomplished by rotating the lever in one direction, while down-shifting is accomplished by rotating the lever in the opposite direction. 

\subsection{Neutral Sensing}

A \emph{neutral sense} wire is attached to the shift drum. The wire is grounded when the transmission is in neutral. The wire is left floating when the vehicle is in gear. 

\subsection{Gear Position Sensing}

The stock transmission has no means of determining what gear the transmission is in; it can only determine when the transmission is in neutral using the aforementioned neutral sense wire. The ECU has provisions for sensing gear position by reading a geared potentiometer that can be attached to the shift drum, however this requires mechanical modification of the shift drum itself.

\subsection{Previous Implementations and Shortcomings}

There are two major downfalls to operating a purely manual transmission: it requires a great deal of dexterity and effort on behalf of the driver, and shifting time is relatively slow (around a second) compared to an electronically-assisted transmission (under one-hundred milliseconds.) 

For these reasons, the Formula SAE vehicle switched to an electronically-assisted transmission system in the 2009 competition year. This implementation replaced a purely mechanical clutch pedal and gear shifter with electropneumatic actuators linked directly to the clutch and gear selector levers. Driver control was achieved with a set of paddles located on the steering wheel: an up-shift paddle and a down-shift paddle.

The electropneumatic system relieved the driver from the burden of manually engaging the clutch. It also reduced shift-time by a factor of nearly ten. However, the system made use of binary pneumatic valves which reduced the ability to engage the clutch smoothly. The rate at which the clutch was engaged was limited by the air flow rate of the valve itself, resulting in a rough shifting sequence that was mechanically hard on the transmission. 

Another limitation was posed by the lack of gear position sensing in the stock transmission. Although the system was able to determine when the transmission was in neutral, there was no way to determine if the gears had meshed after up- or down-shifting. To do so would require modifying the shift drum to incorperate a gear-sensing potentiometer. This led to situations where the transmission controller became confused as to which gear was actually selected, and required driver intervention to resynchronize the controller with the transmission.

