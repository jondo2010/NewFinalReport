\section{Electropneumatic Implementation}

\subsection{Simulation}

The physical behaviour 
Several steps were taken in the implementation of the pneumatic system in order to facilitate

A Simulink model was developed of the electropneumatic actuation system in order to verify that fundamental design would work and to gain further insight into the operation of the pneumatic system.

The top level of the Simulink model is shown in \ref{fig:pneumatics_top_level}: A PID controller block is used with closed-loop feedback, and the response to a fixed step input is displayed on the scope block.

In modelling the electropneumatic system, 3 subsystems were identified and modelled independently:

\begin{enumerate}
  \item the PWM generator;
  \item the solenoid valves;
  \item and the pneumatic actuator (single-acting spring-return cylinder).
\end{enumerate}

\begin{figure}[h]
\centering
\includegraphics[scale=1]{implementation/figures/pneumatic_modelling1.eps}
\caption{Top-level electropneumatic model for simulation in Simulink.}
\label{fig:pneumatics_top_level}
\end{figure}

\paragraph{PWM Generator}

The PWM generator in Simulink was built using the instantaneous model presented by \citet{valve_models} which compares a generated saw-tooth signal, $V_{saw}$ with a period $T_{saw}$, with the input signal $V_{in}$ to obtain the pulse-width-modulated signal $U$:

\begin{equation}
\label{eq:pwm_generation}
U\left(t\right) = 
\begin{cases}
1 & V_{in}\left(t\right) \geq V_{saw}\left(t\right) \\
0 & V_{in}\left(t\right) < \left(t\right)
\end{cases}
\end{equation}

The Simulink model which implements Eq. \ref{eq:pwm_generation} can be seen in Fig. \ref{fig:pneumatics_pwm}. The input to the subsystem, shown as \emph{In1}, is $V_{in}$, and the output, shown as \emph{Out1} is the pulse width modulated signal $U$. A Saturation and block was added to limit the input signal $V_{in}$ to the range [0..1]. A dead zone block was added to simulate the effect of dead zone in the response of the solenoid valves to a PWM signal: if the pulse width is too short, the current through the solenoid cannot generate enough force to open the poppet, so the valve stays shut. This is a parameter of the solenoid valve, and was quantified experimentally by \cite{valve_models} as the minimum input signal $V_{in}$ required to open the valve at all. \Citet{accurate_position} also account for a minimum possible duty cycle in the solenoid valve input signal as

\begin{equation}
  \label{eq:pwm_duty_min}
  d_{min}=\left(T_{vr}/T_{PWM}\right)\cdot100\%
\end{equation}

where $T_{vr}$ is the valve response time, and $T_{PWM}$ the PWM period.


\begin{figure}[h]
\centering
\includegraphics[scale=1]{implementation/figures/pneumatic_modelling2.eps}
\caption{Simulink PWM Generator model.}
\label{fig:pneumatics_pwm}
\end{figure}

\paragraph{Solenoid Valves}



\paragraph{Pneumatic Actuator}

The Simulink pneumatic actuator subsystem can be seen in Fig. \ref{fig:pneumatics_actuator}. In order to save time, pre-built Simulink blocks from the Simscape package were used to model the dynamics of the actuator. Physical port 1 (denoted by the octagonal port symbol) in \ref{fig:pneumatics_actuator} is the air inlet. Physical ports 2 and 3 are the displacement ports of the cylinder, and regular port 1 is used to display the displacement on a scope.

\begin{figure}[h]
\centering
\includegraphics[scale=1]{implementation/figures/pneumatic_modelling3}
\caption{Simulink actuator model.}
\label{fig:pneumatics_actuator}
\end{figure}

\paragraph{Overall Electropneumatic Simulink Model}

\begin{figure}[h]
\centering
\includegraphics[scale=0.65]{implementation/figures/pneumatic_modelling4}
\caption{Simulink electropneumatic model.}
\label{fig:pneumatics_model}
\end{figure}

\subsection{Solenoid Valves}

Solenoid valves and cylinders from SMC corp were chosen for a physical implementation of the design. 

\begin{table}[H]
  \caption{Solenoid valve specifications.\label{tab:xbee_commands}}
  \centering

  \begin{tabular}{|l|l|}
  \hline
  Part & VQZ115-6L1-N1-PR \tabularnewline
  \hline
  Coil Voltage & \unit{12}{\volt} \tabularnewline
  \hline
  Configuration & 3-port normally closed \tabularnewline
  \hline
  Flow coefficient & $C_v=\unit{0.23}{}$ \tabularnewline
  \hline
  Max. Operating Frequency & \unit{20}{\hertz} \tabularnewline
  \hline
  Max. Pressure & \unit{0.7}{\mega\pascal} \tabularnewline
  \hline
  \end{tabular}
\end{table}

\subsection{Pneumatic Actuators}


\subsection{Positional Feedback Sensors}


\subsection{Mechanical Linkage}
