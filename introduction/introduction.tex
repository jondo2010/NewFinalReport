\chapter{Introduction\label{cha:introduction}}

The purpose of this chapter is to provide a broad introduction to the project by first introducing our student group, our motivations for undertaking the project, the scope and definition of the problems we would like to solve, and the strategy undertaken to solve these problems. We will also close with a brief outline of each section in this report.

\section{Formula SAE}

Formula SAE\nomenclature{SAE}{Society of Automotive Engineers} is an engineering student design competition organized by the Society of Automotive Engineers dating back to 1978 \cite{fsaehistory}. Students from the University of Manitoba have participated in the competition almost every year since 1985. The competition consists of designing and constructing a small, open-wheeled, formula-style race car.

The Formula SAE vehicle is a performance car built with the primary goal of doing well in the dynamic events at the yearly competitions. These events test the vehicles' abilities in acceleration, braking, and handling. 

\section{Motivation}

Many of the issues that directly affect the teams' performance at competition relate to driver training, feedback, and the tunability of the car. Most of the mechanical systems on the car must currently be imprecisely hand-tuned, and are packaged in hard to reach places, and require body panels or the seat to be removed for access.

Our overall goal is thus to improve the precision, adjustability, and repeatability of adjustment, of various important mechanical systems in the car, and to improve the efficiency of testing. Highly desirable is a shorter driver-feedback-tuning loop in order to eliminate overshoot and undershoot in tuning, and to avoid other external disturbances.

A second major goal is to improve upon the transmission control systems of previous years' designs.

\section{Problem Definition}

The goal of this work is to tackle several distinct issues on the 2010 Formula SAE Vehicle, specifically:

\begin{itemize}
 \item to improve the electronic control of the transmission;
 \item to introduce an automatically-adjusting variable-length air intake;
 \item to make the brake bias adjustable electronically;
 \item to broadcast telemetry data from on-board systems to the pit crew; and
 \item to create an easy-to-use driver interface.
\end{itemize}

Improving the electronic control of the transmission will decrease the time required to change gears, improve the mechanical reliability of the transmission, and decrease the effort required by the driver to change gears. 

Introducing an automatically adjusting variable-length air intake will widen the peak torque output from the engine. This will improve engine responsiveness and vehicle acceleration.
 
Making the brake bias adjustable electronically will allow on-the-fly bias adjustment, even while the vehicle is in motion. This will eliminate the need for manual adjustment and calibration.
 
Broadcasting telemetry data to the pit crew will allow the team to make decisions regarding the dynamic parameters of the vehicle. This will optimize vehicle performance and provide feedback to the crew regarding the quality of any parameter tuning.

Creating an easy-to-use driver interface will allow the driver to tune the dynamic parameters of the vehicle from a simple interface located on the steering wheel, and will also provide the driver with real-time feedback from the various vehicle sensors. This will allow the driver to optimize the vehicle's dynamic parameters for their own driving style, and keep the driver informed of the vehicle's state without being distracting.
 
\section{Strategy \label{sec:intro_strategy}}

Four networked electronic modules and an electro-pneumatic actuation system will be created to meet the goals outlined in the problem definition. The four modules are:

\begin{itemize}
\item the transmission and engine control module;
\item the brake bias adjustment module;
\item the wireless telemetry module; and
\item the driver interface module.
\end{itemize}

The transmission and engine control module will interface with the electro-pneumatic actuation system to control the transmission. The design and implementation of the system requires several steps:

\begin{enumerate}

\item Absolute requirements and specifications for system parameters are first established. The electromechanical interface linking the control systems and mechanical systems are decided between our group and the mechanical engineers responsible for each respective system. The measurement requirements from the mechanical systems are established and the appropriate sensors are chosen.

\item Appropriate components for the electronic modules are chosen, and the circuits are designed. The schematic of each module is designed and an appropriate printed circuit board is laid out. The boards are manufactured and populated with components. Each board must be tested to ensure all wiring is correct and the functionality of all the components is correct.

\item Firmware for each module must be implemented once the hardware design is finished. Software libraries for various components are written and tested. The libraries are combined with a high-level control algorithm for each module.
 
\item Each module is tested in isolation to ensure proper functionality. 

\item The modules are then interconnected and tested again.

\end{enumerate}

\section{Outline of Thesis}

This thesis is organized into 8 chapters. After this introductory chapter, Ch. \ref{cha:background} provides background information describing the vehicle systems we will interact with, the problems associated with these systems we are attempting to solve, and a description of previous efforts to solve these problems. 

In Ch. \ref{cha:goals} the goals and requirements of the design work will be specified for each of the areas described in Ch. \ref{cha:background}. 

Chapter \ref{cha:design} describes our design to meet these requirements. 

Chapter \ref{cha:implementation} describes our implementation of the design.
