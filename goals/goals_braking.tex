\section{Braking}

The most desired improvement for the braking system is to eliminate the need for adjusting the brake bias manually, which requires substantial effort and time on behalf of the pit crew. Thus, the main goal for the braking system is to introduce an electronically-adjustable bias system. As the braking system is subject to mechanical wear and inconsistency, a secondary goal is to provide the ability to recalibrate the braking system at will.

\subsection{Bias Adjustment}

Eliminating the need to manually adjust the bias will improve the tunability of the vehicle. Allowing the driver to adjust the bias while on the track means the driver can adjust for varying track conditions like rain. This adjustment must occur quickly, because the driver may not apply the brakes during the adjustment.

\begin{itemize}

\item Allow the driver to adjust the brake bias from a control in the cockpit.
\item Provide a front-rear adjustment range of 30/70 to 70/30, with a resolution of $0.5\%$.
\item The bias must be able to be adjusted from one limit to the other in less than \unit{1}{\second}.
\item The bias must be adjustable while the vehicle is in motion.
\item The bias must not be adjusted while the brakes are being applied.

\end{itemize}

\subsection{Bias Calibration}

Because the system is subject to wear and inconsistency, the crew should be able to re-centre the balance bar as needed. This should be a short procedure, as there is usually not much time between events at a competition.

\begin{itemize}

\item Provide the ability to recalibrate the bias adjustment system to account for mechanical wear and inconsistency. 
\item It must take less than 30 seconds to complete the bias adjustment calibration.

\end{itemize}
