\section{Driver Interface}

\subsection{Overview}

The driver interface consists of all the controls available to the driver to change the performance dynamics of the vehicle, all of the warning indicators that inform the driver of potentially hazardous situations, and all of the diagnostic information available to advise the driver on the overall condition of the vehicle.

An ideal driver interface is unobtrusive and requires as little driver attention and effort as possible. Controls required while the vehicle is in motion should be operable without the driver needing to divert their attention from the track. Warnings should be highly visible and discernible with minimal effort. Diagnostic information should not clutter or overwhelm the driver. Tuneable vehicle parameters should be presented to the pit crew with an intuitive interface that cannot be accidentally triggered by the driver.

\subsection{Transmission Control}

As discussed extensively in Sec. \ref{sec:background_transmission}, the driver must be able to up- and down-shift as quickly as possible. Early Formula SAE vehicles relied on a purely mechanical clutch and shift lever configuration. In 2008, the mechanical system was replaced with a pair of paddle shifters located on the steering wheel. Although the system being controlled by the paddles has evolved, the actual interface has remained constant for the past two incarnations.

The 2009 Formula SAE vehicle has feature known as \emph{neutral find}, which allows the driver to quickly shift the vehicle into neutral at the touch of a button. Normally, shifting into neutral would require the driver to down-shift until they reach neutral. This feature is especially useful for preventing the vehicle when leaving the track and entering the pit area. 

Another innovation of 2009 was the introduction of \emph{launch control}, which enabled the driver to accelerate rapidly from a stand-still without having to shift from neutral to first gear. The sensitivity of the engine means it is very easy to stall the motor if the clutch is engaged too quickly, especially when transitioning into first gear. Launch control eliminates this by monitoring the relative speeds of the engine and wheels, and modulates the clutch to provide the most power and traction to the wheels without stalling the engine. 
 
\subsection{Warning Indicators}

The driver must be alert to the possibility of the engine overheating, or of any drops in oil pressure that could cause damage to the engine. Previous implementations of the driver interface provided these warnings with simple LEDs that interfaced with the water temperature and oil pressure sensors. Although simple, the indicators were unreliable and would sometimes neglect to function, or would trigger when no problem was actually present. 

Several other warning indicators, such as a low voltage warning or an up-shift indicator, would be beneficial to the driver but have as of yet been out of scope for the Formula SAE team.

\subsection{Vehicle Diagnostics}

Diagnostic information about the vehicle, such as the state of the telemetry system, current gear, engine RPM, and so on have not been included in previous implementations. These diagnostics would be beneficial to the driver and the pit crew, but have been so far out of the scope of the previous implementations.

