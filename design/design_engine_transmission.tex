\section{Engine and Transmission Module}

The engine and transmission module provides optimized selection of the variable-length intake, changes gears at the driver's request without requiring manual clutching, and provides transmission features that make driving easier. Figure \ref{fig:design_engine_overview_block} shows an overview of the engine and transmission module and it's interactions with the environment.

\begin{figure}[H]
	\centering
 	\begin{tikzpicture}[auto, node distance=2cm, draw=black!70, >=stealth', font=\footnotesize]
  \node [block, blue shiny, minimum width=4cm, text width=4cm, right=-1cm, node distance=1.5cm] (module) {Braking Module};

  \node at ($(module.east)+(1cm,0)$) [block, text width=0.8cm] (ecu) {ECU};

  \node [bus, above of=module, node distance=1cm] (bus1) {};
  \node [bus, above of=ecu, node distance=1cm] (bus2) {};

  \draw [<->, thick] (ecu) to (module);

  \draw [-, line width=3pt] (bus1) --  ++(-1cm,0);
  \draw [-, line width=3pt] (bus1) -- node[label=above:CAN Bus]{} (bus2) -- ++(1cm, 0);
  \draw [<->, thick] (module) -- (bus1);
  \draw [<->, thick] (ecu) -- (bus2);

  \node at ($(module.west)!0.5!(module.south)+(0,-1.25cm)$) [red shiny, circle, label={[text width=1.5cm, rotate=90, below=5pt, left=10pt]below right:Servo}] (motor) {M};
  \node [block, left of=motor, node distance=1.5cm, text width=1cm] (intake) {Intake};

  \draw [<-, thick] (intake) -- (motor);
  \draw [<-, thick] (motor) -- ($(module.south west)!(motor.north)!(module.south)$);

  \node [block, below of=module, inner xsep=0pt, right=0cm, node distance=1.5cm] (pneumatics) {Electro-pneumatic system};
  \draw [<-, thick] (pneumatics) -- ($(module.south west)!(pneumatics.north)!(module.south)$);

  \node [block, right of=pneumatics, text width=2cm, node distance=3cm, above=0.25cm] (shift) {Shift lever};
  \node [block, right of=pneumatics, text width=2cm, node distance=3cm, below=0.25cm] (clutch) {Clutch lever};

  \draw [<-, thick, dashed] (clutch) -- ($(pneumatics.north east)!(clutch.west)!(pneumatics.south east)$);
  \draw [<-, thick, dashed] (shift) -- ($(pneumatics.north east)!(shift.west)!(pneumatics.south east)$);

  %%% Legend

  \draw [->, thick] ($(shift.east)+(2cm,0cm)$) -- ++(0.5cm,0) node [label={[font=\tiny]below:Electrical}] {} -- ++(0.5cm,0);
  \draw [->, thick, dashed] ($(clutch.east)+(2cm,0cm)$) -- ++(0.5cm,0) node [label={[font=\tiny]below:Mechanical}] {} -- ++ (0.5cm,0);
\end{tikzpicture}

	\caption{An overview of the engine and transmission module and it's environmental interactions.}
	\label{fig:design_engine_overview_block}
\end{figure}

The intake runner position is mechanically actuated with a \emph{servo motor}. The module generates the control signals required by electro-pneumatic system to actuate both the clutch and shift levers. The ECU interfaces with the module both directly through discrete inputs and through the CAN bus.

\subsection{Driver-Initiated Processes}

The major processes of the engine and transmission module are \emph{up-shifting}, \emph{down-shifting}, and \emph{neutral find}. They are described at a high level using flow charts. Requests to perform these processes are broadcast over the network from the driver interface. The module listens for these requests and executes the correct process accordingly.

\subsubsection{Up-Shifting}

The up-shift procedure is described in \ref{fig:transmission_upshift_flow}. Up-shifting does not require disengaging the clutch, merely a small decrease in engine RPM. The ECU provides a shift-cut feature to cut spark to the engine during a shift operation to avoid the driver having to manually modulate the throttle during a shift.

\begin{figure}[H]
	\centering
	\begin{tikzpicture}[auto, node distance=2cm, draw=black!70, >=stealth', font=\scriptsize]
  \node[start, text width=1.2cm] (start) {Upshift Request};
  \node [decision, right of=start, text width=1cm, inner sep=0pt, node distance=2.5cm] (at_top) {Top\\gear?};
  \node [end, below of=at_top] (done) {Done};
  \node [block, right of=at_top, text width=1.5cm, node distance=2.5cm] (shiftcut) {Enable shiftcut};

  \node [block, right of=shiftcut, text width=1.5cm, node distance=2.5cm] (wait_shiftcut) {Wait for RPM drop below $RPM^{TH}_{cut}$};
  \node [block, right of=wait_shiftcut, text width=1.5cm, node distance=2.5cm] (upshift) {Engage upshift solenoid};
  \node [block, right of=upshift, text width=1.5cm, node distance=2.5cm] (wait_upshift) {Wait for shift feedback};
  \node [block, below of=wait_upshift, text width=1.5cm] (upshift_off) {Disengage upshift solenoid};
  \node [block, left of=upshift_off, text width=1.5cm, node distance=2.5cm] (shiftcut_off) {Disable shiftcut};

  \draw [->, thick] (start) -- (at_top);
  \draw [->, thick] (at_top) -- node[]{yes} (done);
  \draw [->, thick] (at_top) -- node[]{no} (shiftcut);

  \draw [->, thick] (shiftcut) -- (wait_shiftcut);
  \draw [->, thick] (wait_shiftcut) -- (upshift);
  \draw [->, thick] (upshift) -- (wait_upshift);
  \draw [->, thick] (wait_upshift) -- (upshift_off);
  \draw [->, thick] (upshift_off) -- (shiftcut_off);
  \draw [->, thick] (shiftcut_off) -- (done);
\end{tikzpicture}

	\caption{The transmission up-shift procedure.}
	\label{fig:transmission_upshift_flow}
\end{figure}

The up-shift process depends upon engine RPM and gear position, both of which are provided by the ECU. The shift-cut feature is engaged on the ECU to drop the engine RPM by a small amount known as the cut RPM threshold, or $RPM^{TH}_{cut}$. The exact value of $RPM^{TH}_{cut}$ can be tuned on the ECU. Once the RPM reaches the cut threshold, the upshift solenoid is engaged, which actuates the pneumatics to push the shift lever. Once the module receives notification that the current gear has incremented, the pneumatics are relaxed, and shift cut feature is disengaged.

\nomenclature{$RPM^{TH}_{cut}$}{The threshold RPM is expected to drop when shift-cut is engaged for no-lift up-shifting.}

\subsubsection{Down-Shifting}

The down-shift procedure is described in \ref{fig:transmission_downshift_flow}. The downshift procedure differs from the upshift procedure in that the downshift requires use of the clutch. 

\begin{figure}[H]
	\centering
	\begin{tikzpicture}[auto, node distance=2cm, draw=black!70, >=stealth', font=\scriptsize]
  \node[start, text width=1.4cm] (start) {Downshift Request};
  \node [decision, right of=start, text width=1cm, inner sep=0pt, node distance=2.5cm] (in_neutral) {Neutral?};
  \node [end, below of=in_neutral] (done) {Done};
  \node [block, right of=in_neutral, text width=1.5cm, node distance=2.5cm] (clutchin) {Disengage clutch};

  \node [block, right of=clutchin, text width=1.5cm, node distance=2.5cm] (downshift) {Engage downshift solenoid};
  \node [block, right of=downshift, text width=1.5cm, node distance=2.5cm] (wait_downshift) {Wait for shift feedback};
  \node [block, right of=wait_downshift, text width=1.5cm, node distance=2.5cm] (downshift_off) {Disengage downshift solenoid};
  \node [block, below of=downshift_off, text width=1.5cm] (wait_clutchin) {Wait for clutch-in request};
  \node [block, left of=wait_clutchin, text width=1.5cm, node distance=2.5cm] (clutchout) {Engage clutch};

  \draw [->, thick] (start) -- (in_neutral);
  \draw [->, thick] (in_neutral) -- node[]{yes} (done);
  \draw [->, thick] (in_neutral) -- node[]{no} (clutchin);

  \draw [->, thick] (clutchin) -- (downshift);
  \draw [->, thick] (downshift) -- (wait_downshift);
  \draw [->, thick] (wait_downshift) -- (downshift_off);
  \draw [->, thick] (downshift_off) -- (wait_clutchin);
  \draw [->, thick] (wait_clutchin) -- (clutchout);

  \draw [->, thick] (clutchout) -- (done);
\end{tikzpicture}

	\caption{The transmission down-shift procedure.}
	\label{fig:transmission_downshift_flow}
\end{figure}

From the drivers perspective, down-shifting is a two-stage process:

\begin{enumerate}
  \item The driver pulls and holds the down-shift paddle on the steering wheel to request the gear change; and
  \item The driver releases the down-shift paddle to requests the clutch re-engage.
\end{enumerate}

Adding this extra clutch engagement request allows the driver to delay re-engagement of the clutch so that they can blip the throttle to avoid engine compression and the possibility of a spinout when the clutch plates re-engage\footnote{"Blipping" was previously described in Sec. \ref{sec:background_transmission}.}. 

Once the driver requests a down-shift, the clutch is disengaged and the down-shift solenoid is engaged to change gears. Once the correct gear has engaged, the solenoid is disengaged and the controller waits for the driver to release the down-shift paddle. When the paddle is released, the clutch is re-engaged. A minimum clutch disengagement time $t_{clutch_{min}}$ is observed in the event the driver pulls the paddle and releases it without any delay. This parameter is tunable.

\nomenclature{$t^{clutch}_{min}$}{Minimum clutch disengagement time during a downshift request.}

\subsubsection{Neutral Find}

Neutral find can be used by the driver to quickly shift into neutral when entering the pit area, without having to manually cycle through gears. The procedure is described in \ref{fig:transmission_neutralfind_flow}.

\begin{figure}[H]
	\centering
	\begin{tikzpicture}[auto, node distance=2cm, draw=black!70, >=stealth', font=\scriptsize]
  \node[start, text width=1.4cm] (start) {Start};
  \node [decision, right of=start, text width=1.2cm, inner sep=0pt, node distance=2.5cm] (in_neutral1) {Neutral?};
  \node [end, below of=start, node distance=1.5cm] (done) {Done};
  \node [block, right of=in_neutral1, text width=1.5cm, node distance=2.5cm] (clutchin) {Disengage clutch};
  \node [block, right of=clutchin, text width=1.5cm, node distance=2.5cm] (downshift) {Downshift};
  \node [decision, right of=downshift, text width=1.2cm, inner sep=0pt, node distance=2.5cm] (in_neutral2) {Neutral?};

  \node [block, below of=clutchin, text width=1.5cm, node distance=1.5cm] (clutchout) {Engage clutch};

  \draw [->, thick] (in_neutral1.south) -- node[]{yes} (in_neutral1.south |- clutchout);

  \draw [->, thick] (start) -- (in_neutral1);
  \draw [->, thick] (in_neutral1) -- node[]{no} (clutchin);
  \draw [->, thick] (clutchin) -- node[coordinate](x1){} (downshift);
  \draw [->, thick] (downshift) -- (in_neutral2);
  \draw [->, thick] (in_neutral2.north) -- node[above]{no} ++(0,0.25cm) -| (x1);

  \draw [->, thick] (in_neutral2.south) |- node[]{yes} (clutchout);
  \draw [->, thick] (clutchout) -- (done);
\end{tikzpicture}

	\caption{The transmission neutral find procedure.}
	\label{fig:transmission_neutralfind_flow}
\end{figure}

When the driver engages the neutral find feature, the transmission controller disengages the clutch and downshifts until the neutral sensor indicates the transmission has reached the neutral gear. The clutch lever is then released with the transmission in neutral. If the driver attempts to engage the feature while the vehicle is already in neutral, the procedure is aborted.

\subsection{Variable Intake}

The length of the intake runners is modulated automatically to provide the optimal torque output for the current engine RPM. The control flow chart is shown in Fig. \ref{fig:engine_varintake_flow}. 
\begin{figure}[H]
	\centering
	\begin{tikzpicture}[auto, node distance=2cm, draw=black!70, >=stealth', font=\scriptsize]
  \node [start, text width=1.4cm] (start) {Start};
  \node [block, right of=start, text width=1.5cm, node distance=2.5cm] (update) {Update RPM};
  \node [decision, right of=update, text width=1cm, inner sep=0pt, node distance=2.5cm] (ideal) {Ideal length?};

  \node [block, right of=ideal, text width=1.5cm, node distance=2.5cm] (switch) {Switch lengths};
  \node [block, right of=switch, text width=1.5cm, node distance=2.5cm] (wait) {Adjust crossover};

  \draw [->, thick] (start) -- (update);
  \draw [->, thick] (update) -- (ideal);
  \draw [->, thick] (ideal) -- node[above]{no} (switch);
  \draw [->, thick] (switch) -- (wait);
  \draw [->, thick] (wait) -- ++(0, -1.5cm) node[coordinate](x1){} -| ($(start.east)!0.5!(update.west)$);
  \draw [->, thick] (ideal) -- node[]{yes} (ideal.south |- x1);

\end{tikzpicture}
	\caption{The variable intake adjustment flow.}
	\label{fig:engine_varintake_flow}
\end{figure}

Optimizing the runner length requires coordination with the ECU to acquire the latest RPM values. A set-point RPM value to be determined will act as a cross-over point at which the runner length changes. As the engine RPM crosses over this line, the runner length is modulated. A brief timeout period exists to prevent the runner length from oscillating if the driver is operating near the cross-over point.

\subsection{Advanced Transmission Features}

The transmission portion of the module implements several features to aide the driver. The driver can enable or disable these features from the driver interface. The module will listen for feature requests over the network and act accordingly.

\subsubsection{Auto Up-shift}

This feature of the engine module is aimed primarily at improving performance in the acceleration event. Based on known torque curves, a table of optimal shift points in the RPM range is developed. As the engine reaches the top RPM for a given gear, the engine module will automatically upshift to the next gear, without any driver input. All the driver needs to do is maintain full throttle, and hang on.

\subsubsection{Launch}

Full-throttle launch uses an important feature of the ECU called launch control. From a stand-still, the slip ratio of the driven wheels to the non-driven wheels is monitored, and the engine output power is reduced until the ratio reaches 1:1. Drivers use this feature by maintaining full-throttle at the starting line while holding the brake pedal. As soon as the brake pedal is released, the the engine module will release the clutch in a controlled manner in an attempt to get the best possible acceleration.

\subsubsection{Crawl}

Part-throttle launch is a feature designed to mimic an automatic transmission. By controlling the clutch position, and thereby modulating the amount of torque transferred to the wheels for a short period of time, the car can be made to creep slowly from a stand-still. This will be used when driving up to the starting line of various dynamic events.

\subsection{Hardware}

Show system hardware diagram

\subsubsection{Microcontroller}

Execute system control software.

In-circuit programmable and debuggable.

Has built-in CAN controller.

Has built-in RAM and ROM as well as EEPROM for holding configuration parameters.

\subsubsection{CAN transceiver}

Interface module with the CAN bus.

Be capable of terminating the bus.

\subsubsection{High current solenoid drivers}

Controlling pneumatic solenoid valves (4)
  
Controlling starter solenoid

\subsubsection{I/O lines to the ECU}

The engine and transmission module has a several CMOS-level outputs to discrete control pins on the ECU:

\begin{itemize}
  \item A line to the launch control pin, which when toggled enables and disables the launch control feature,
  \item A line to the shift cut pin, which when held high enables shift cut, decreasing engine power to allow for an upshift,
  \item A line to the traction cut percent pin, which is a \unit{0-5}{\volt} level signalled input changing the amount of acceptable wheel slip before traction control cuts in.
  \item A line to the traction control on/off pin, which toggles the traction control feature,
  \item A line to the traction control wet/dry pin, which swaps between two traction control presets.
\end{itemize}

The outputs of these lines are controlled in the system software through the event scheduler. 

\subsection{Software}

Software overview diagram

\subsubsection{Transmission Manager}

Listens to transmission requests from the driver over the network.

Uses the event scheduler to sequence the complex series of actuation vectors required by upshifting and downshifting.

Implement the PID feedback controller used to actuate the pneumatic system.

Interacts with the PWM generator to create the control signals required to actuate the pneumatic system.


\subsubsection{Intake Manager}

Continously monitor engine RPM and throttle position through messages from the ECU.

Will adjust intake runner length based on a functional map of RPM and throttle position.

Use hysterisis to avoid instability.

Map will be generated through dynometer testing.


\subsubsection{Starter Manager}

Listen for driver requests to start the engine.

Provide a means of one-touch starting -- sequences the actual starting of the engine through the solenoid driver. 


\subsubsection{Event Scheduler}

Allow for complex sequencing of events in time.

Controllers may schedule new events to occur at various points in time.

Scheduler will continuously update the schedule and signal the main control loop to execute events that are now current.


\subsubsection{CAN Interface}

Provide a means of interfacing with the physical bus. 

Allow direction of particular messages to particular modules.


\subsubsection{PWM Generator}

Generate PWM signals of at least 20 Hz with a resolution of 1\% duty
cycle.

Two-channel design


\subsubsection{Main Control Loop}

Initializes the system and brings into a known state.

Waits for pending events to be executed and executes them.

Monitor for system faults and react accordingly.

