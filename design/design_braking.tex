\section{Braking Module\label{sec:Braking-Module-Design}}

The braking module adjusts the brake bias electronically as requested by the driver. It also provides the ability to calibrate itself to account for mechanical wear and to account for a degree of component tolerance that may cause unwanted rotation of the balance bar. Figure \ref{fig:design_brake_overview_block} shows an overview of the braking module and its interactions with the environment. 

\begin{figure}[H]
\centering
\begin{tikzpicture}[auto, node distance=2cm, draw=black!70, >=stealth', font=\footnotesize]
  \node [block, minimum width=2cm, inner xsep=0] (pressure1) {Front Pressure Sensor};
  \node [block, minimum width=2cm, inner xsep=0, right of=pressure1, node distance=4cm] (pressure2) {Front Pressure Sensor};

  \node [block, blue shiny, minimum width=6cm, text width=4cm, below of=pressure1, right=-1cm, node distance=1.5cm] (module) {Braking Module};

  \node [block, below of=pressure1, right=-1cm, node distance=3cm, text width=1.1cm] (eot1) {Left EOT Switch};
  \node [block, below of=pressure2, left=-1cm, node distance=3cm, text width=1.1cm] (eot2) {Right EOT Switch};

  \node at ($(eot1)!0.5!(eot2)$) [red shiny, circle, label={[text width=1.5cm, rotate=90, below=5pt, left=10pt]below right:Stepper Motor}] (motor) {M};

  \node [rectangle, below of=motor, minimum width=2cm, label=below:Bias Bar, pattern=north east lines, node distance=1cm] (bias) {};

  \draw [->, thick] (pressure1.south) to ($(module.north west)!(pressure1.south)!(module.north east)$);
  \draw [->, thick] (pressure2.south) to ($(module.north west)!(pressure2.south)!(module.north east)$);

  \draw [->, thick] (eot1.north) to ($(module.south west)!(eot1.north)!(module.south east)$);
  \draw [->, thick] (eot2.north) to ($(module.south west)!(eot2.north)!(module.south east)$);

  \draw [<-, thick] (motor.north) to ($(module.south west)!(motor.north)!(module.south east)$);
  \draw [->, thick, dashed] (motor.south) to ($(bias.north west)!(motor.south)!(bias.north east)$);

  \draw [->, thick, dashed] (bias) -| (eot1);
  \draw [->, thick, dashed] (bias) -| (eot2);

  %%% CAN Bus

  \node [bus, name=can1, right of=module, label={[rotate=90, left=0.75cm]below right:CAN Bus}, node distance=3.5cm] {CAN Bus};
  \node [bus, name=can2, below of=can1, node distance=1cm] {};
  \node [bus, name=can3, above of=can1, node distance=1cm] {};

  \draw [<->, thick] (module) to (can1);
  \draw [-, line width=3pt] (can1) -- (can2);
  \draw [-, line width=3pt] (can1) -- (can3);

  %%% Legend

  \draw [->, thick] ($(eot2.east)+(2cm,0cm)$) -- ++(0.5cm,0) node [label={[font=\tiny]below:Electrical}] {} -- ++(0.5cm,0);
  \draw [->, thick, dashed] ($(eot2.south east)+(2cm,0cm)$) -- ++(0.5cm,0) node [label={[font=\tiny]below:Mechanical}] {} -- ++ (0.5cm,0);
\end{tikzpicture}
\caption{Overview of the braking module and it's environment.}
\label{fig:design_brake_overview_block}
\end{figure}

\nomenclature{EOT}{End of Travel}
A \emph{stepper motor} is connected to the balance bar and used to rotate it to the desired position. The motor should be bi-directional. Two \emph{end-of-travel} (EOT) switches are used to determine when the balance bar is at its end of travel. \emph{Pressure sensors} on the front and rear braking cylinders are used to determine the current pressure in each system. These components were chosen as part of the mechanical design of the brake pedal assembly by another team member. The actual choice of stepper motor has yet to be determined.

\subsection{Driver-Initiated Processes \label{sec:braking_processes}}

The major processes of the braking module are \emph{brake bias adjustment}, \emph{brake bias calibration}, and \emph{pressure calibration}. Requests to perform either of the processes are broadcast over the network from the driver interface. The module listens for these requests and executes the correct process accordingly. 

\subsubsection{Brake Bias Adjustment}

The brake bias adjustment procedure is shown in Fig. \ref{fig:design-braking-bias-adjustment}. The module performs a series of safety checks before rotating the balance bar to adjust the relative force distribution between the two brake master cylinders.

\begin{figure}[H]
	\centering
	\begin{tikzpicture}[auto, node distance=2cm, draw=black!70, >=stealth', font=\scriptsize]
  \node [start] (start) {Start};
  \node [decision, right of=start, text width=1cm, inner sep=0pt] (valid_bias) {Valid Bias?};
  \node [end, below of=start, node distance=1cm] (abort) {Abort};

  \node [block, right of=valid_bias, text width=1.5cm, inner xsep=0pt, node distance=2.5cm] (calc) {Calculate Offset Steps};
  \node [block, right of=calc, text width=1.5cm, inner xsep=0pt, node distance=2.5cm] (step) {Step by offset};
  \node [block, right of=step, text width=1cm, inner xsep=0pt] (store) {Store new offset};
  \node [end, right of=store] (done) {Done};

  \draw [->, thick] (start) to (valid_bias);
  \draw [->, thick] (valid_bias.south) |- node[below]{no} (abort);
  \draw [->, thick] (valid_bias) to node[above]{yes} (calc);
  \draw [->, thick] (calc) to (step);
  \draw [->, thick] (step) to (store);
  \draw [->, thick] (store) to (done);
\end{tikzpicture}

	\caption{Flow-chart for the brake bias adjustment procedure.}
	\label{fig:design-braking-bias-adjustment}
\end{figure}

Of prime importance is that the vehicle is not in motion and the brakes are not engaged while the adjustment is underway. The desired bias ratio is also validated before any adjustment is attempted. If the vehicle is stopped, the brakes are not engaged, and the request is valid, the adjustment may proceed. The module calculates the number of steps required to move from the current position to the desired offset and signals the motor to advance this many steps in the correct direction.

\subsubsection{Brake Bias Calibration}

A flow-chart of the brake bias calibration procedure is shown in Fig. \ref{fig:brake_bias_calibration_flow}. By recording the number of steps it takes to rotate the balance bar from through its range of motion, we can determine the correct centre point for the bias bar. 

\begin{figure}[H]
	\centering
	\begin{tikzpicture}[auto, node distance=2cm, draw=black!70, >=stealth', font=\scriptsize]
  \node [start] (start) {Start};

  \node [block, right of=start, text width=1cm, inner xsep=0pt] (step_left) {Step Left};
  \node [decision, right of=step_left, text width=1.0cm, inner sep=0pt] (at_left) {At leftend?};

  \node [block, right of=at_left, text width=1cm, inner xsep=0pt] (step_right) {Step Right};
  \node [block, right of=step_right, text width=1cm, inner xsep=2pt] (count) {Add to count};
  \node [decision, right of=count, text width=1cm, inner sep=0pt] (at_right) {At rightend?};

  \node [block, below of=at_right, text width=2cm, inner xsep=2pt, node distance=2.5cm] (adjust) {Adjust bias to old value};
  \node [block, left of=adjust, text width=1.5cm, inner xsep=2pt, node distance=3cm] (store) {Store calibration};
  \node [end, left of=store, node distance=7cm] (end) {Done};

  \draw [->, thick] (start) to node[coordinate, name=x1]{} (step_left);
  \draw [->, thick] (step_left) to (at_left);
  \draw [->, thick] (at_left.south) -- ++(0,-0.25) node[below]{no} -| (x1);
  \draw [->, thick] (at_left.east) to node[name=x2]{yes} (step_right);
  \draw [->, thick] (step_right) to (count);
  \draw [->, thick] (count) to (at_right);
  \draw [->, thick] (at_right.south) -- ++(0,-0.25) node[below]{no} -| (x2);

  \draw [->, thick] (at_right.east) -- ++(0.5,0) node[above]{yes} |- (adjust);
  \draw [->, thick] (adjust) to (store);
  \draw [->, thick] (store) to (end);
\end{tikzpicture}

	\caption{Flow-chart for the brake bias calibration procedure.}
	\label{fig:brake_bias_calibration_flow}
\end{figure}

The first step in calibrating the brake bias is to rotate the balance bar to its leftmost point of travel, where it pushes on the left end-of-travel switch. This signals the module that the leftmost extreme has been reached. Next, the balance bar is rotated to its rightmost point of travel, until it pushes on the right end-of-travel switch. The module counts the steps required to travel through its range of motion and saves the count in non-volatile parameter storage. Finally, the brake bias ratio is adjusted back to its previous value.

\subsubsection{Pressure Calibration}

The pressure sensor calibration routine is outlined in Fig. \ref{fig:brake_pressure_calibration_flow}. The module keeps track of the minimum and maximum pressures that will be present in both the front and rear systems. This allows it to determine when the brakes are engaged, and to what extent.

\begin{figure}[H]
	\centering
	\begin{tikzpicture}[auto, node distance=2cm, draw=black!70, >=stealth', font=\scriptsize]
  \node [start] (start) {Start};
  \node [decision, right of=start, right=0cm, text width=1.4cm, inner sep=0pt] (min) {Min Applied?};
  \node [block, right of=min, text width=2cm, inner xsep=0pt, node distance=3cm] (sample1) {Sample Pressures};
  \node [decision, right of=sample1, right=0cm, text width=1.4cm, inner sep=0pt] (max) {Max Applied?};
  \node [block, right of=max, text width=2cm, inner xsep=0pt, node distance=3cm] (sample2) {Sample Pressures};

  \node [decision, below of=sample2, text width=1.4cm, inner sep=0pt, node distance=2.5cm] (valid) {Valid samples?};
  \node [block, left of=valid, text width=1.5cm, inner xsep=2pt, node distance=3cm] (save) {Save Calibration};

  \node [end, below of=save] (success) {Success};
  \node [end, below of=valid] (fail) {Fail};

  \draw [->, thick] (start) -- ($(start.east)!0.5!(min.west)$) node[name=x1,coordinate]{} -- (min);
  \draw [->, thick] (min.south) -- ++(0,-0.25cm) node[below]{no} -| (x1);
  \draw [->, thick] (min) to node [] {yes} (sample1);

  \draw [->, thick] (sample1) to node [name=x2]{} (max);
  \draw [->, thick] (max.south) -- ++(0,-0.25cm) node[below]{no} -| (x2);
  \draw [->, thick] (max) to node [] {yes} (sample2);

  \draw [->, thick] (sample2) to (valid);
  \draw [->, thick] (valid) to node [above] {yes} (save);

  \draw [->, thick] (save) to (success);
  \draw [->, thick] (valid.south) to node [] {no} (fail);

\end{tikzpicture}
	\caption{Flow-chart for the brake pressure calibration sequence.}
	\label{fig:brake_pressure_calibration_flow}
\end{figure}

The first step of pressure calibration is to ask the driver to apply minimum pressure. The brake module signals the driver interface to request the driver take their foot off the brake. Once done, the driver interface signals the brake module that minimum pressure has been applied. The brake module then samples the current pressure in both systems. This procedure is repeated for maximum pressure by asking the driver to push on the brake as hard as possible. If the samples are valid, meaning the minimum pressures are less than the maximum pressures and there is a minimum pressure difference between fully-released and fully-applied, the sampled pressures are stored in non-volatile memory.

\subsection{Hardware}

A high-level overview of the module's hardware design is shown in Fig. \ref{fig:brake_hardware_design_block}. Like the other modules, the heart of the brake module is a micro-controller that runs the software necessary to implement all of the required features. 

\begin{figure}[H]
\centering
\begin{tikzpicture}[auto, node distance=2cm, draw=black!70, >=stealth']
%  \draw[help lines] (-3,-5) grid (8,2);

  %%% Pressure Sensors
  \node [block, name=pressure1, font=\scriptsize, text width=1.5cm, inner xsep=0] {Front Pressure Sensor};
  \node [block, name=pressure2, right of=pressure1, font=\scriptsize, text width=1.5cm, inner xsep=0] {Rear Pressure Sensor};

  %%% Balance bar and EOT switches
  \node [block, name=eot1, right of=pressure2, font=\scriptsize, text width=1.5cm, node distance=3cm] {Left EOT Switch};
  \node [block, name=balance, right of=eot1, font=\scriptsize, text width=1.25cm, inner xsep=0] {Balance Bar};
  \node [block, name=eot2, right of=balance, font=\scriptsize, text width=1.5cm] {Right EOT Switch};

  \node [red shiny, circle, below of=balance, name=motor, node distance=1cm, inner sep=1pt] {M};
  \node [block, below of=motor, node distance=1cm, text width=1.5cm, font=\scriptsize] (driver) {Stepper Driver};

  \draw [->, dashed, thick] (balance) -- (eot1);
  \draw [->, dashed, thick] (balance) -- (eot2);
  \draw [->, dashed, thick] (motor) -- (balance);
  \draw [->, thick] (driver) -- (motor);

  \node [block, name=adc1, below of=pressure1, text width=1.5cm] {ADC};
  \node [block, name=adc2, below of=pressure2, text width=1.5cm] {ADC};

  %%% Microcontroller block
  \path ($(adc1.south west)+(0,-1cm)$) node (mu1) {};
  \path ($(adc1.south west -| eot2.south east) + (0,-1cm)$) node (mu2) {};

  \node [block, name=micro, fit=(mu1) (mu2), inner xsep=0, minimum height=1cm] {Microcontroller};

  \draw [->, thick] (pressure1) to (adc1);
  \draw [->, thick] (pressure2) to (adc2);
  \draw [->, thick] (adc1.south) -- ($(micro.north west)!(adc1.south)!(micro.north east)$);
  \draw [->, thick] (adc2.south) -- ($(micro.north west)!(adc2.south)!(micro.north east)$);

  \draw [->, thick] (eot1.south) -- ($(micro.north west)!(eot1.south)!(micro.north east)$);
  \draw [->, thick] (eot2.south) -- ($(micro.north west)!(eot2.south)!(micro.north east)$);

  \draw [<-, thick] (driver.south) -- ($(micro.north west)!(driver.south)!(micro.north east)$);

  %%% CAN Bus

  \node at ($(micro.east)+(2cm,0)$) [block, name=can, node distance=2cm, inner xsep=2pt] {CAN Transceiver};

  \draw [<->, thick] (micro) to (can);

  \node [bus, name=can1, above of=can, label=above:CAN Bus, node distance=2.5cm] {CAN Bus};
  \node [bus, name=can2, left of=can1, node distance=1cm] {};
  \node [bus, name=can3, right of=can1, node distance=1cm] {};

  \draw [-, line width=3pt] (can) -- (can1);
  \draw [-, line width=3pt] (can1) -- (can2);
  \draw [-, line width=3pt] (can1) -- (can3);

  %%% Background

  \begin{pgfonlayer}{background}
    \path (adc1.north west)+(-0.3,0.3) node (a) {};
    \path (can.south east)+(+0.2,-0.2) node (b) {};
    \path[module] (a) rectangle (b);
  \end{pgfonlayer}

  %%% Legend

  \draw [->, thick] ($(micro.south west)+(0.5cm,-0.5cm)$) -- ++(0.5cm,0) node [label={[font=\tiny]below:Electrical}] {} -- ++(0.5cm,0);
  \draw [->, thick, dashed] ($(micro.south west)+(2cm,-0.5cm)$) -- ++(0.5cm,0) node [label={[font=\tiny]below:Mechanical}] {} -- ++ (0.5cm,0);
\end{tikzpicture}

\caption{Block diagram of the braking module hardware.}
\label{fig:brake_hardware_design_block}
\end{figure}

The module communicates over the bus using a CAN transceiver. A pair of analog-to-digital converters (similar to those used in the engine and transmission module) sample the brake pressure sensors. Unique to the brake module is a \emph{stepper motor driver} that provides the signals necessary to drive the stepper motor connected to the balance bar. The micro-controller interfaces to the end-of-travel switches without any extra hardware.

\subsubsection{Stepper Motor Driver}

The stepper motor driver generates the signals required to drive the stepper motor connected to the balance bar. Approximately \unit{150}{mA} of current are required to generate the torque required to spin the balance bar \footnote{This figure was supplied by the team member responsible for the braking pedal assembly.}.

\subsubsection{Analogue-to-Digital Converters (ADC)}

The ADCs sample the output voltage from pressure sensors attached to the front and rear brake master cylinders. The output voltage of the particular sensors used is \unit{0-5}{\volt}, corresponding to \unit{0-10}{\mega\pascal}. The expected maximum brake pressure is somewhere around \unit{7}{\mega\pascal}. To achieve resolution of \unit{100}{\kilo\pascal}, a 10-bit ADC is sufficient. The pressure values are sampled at \unit{10}{\hertz}.

\subsubsection{End-of-Travel Sensing Switches}

Two end-of-travel switches are tripped whenever the balance bar reaches its leftmost or rightmost extremes. The switches are normally high-impedance, and become switched to ground when they are pushed by a metal tab attached to the balance bar assembly. 

\subsection{Software}

A block-diagram overview of the software design is shown in Fig. \ref{fig:brake_software_design_block}. The functionality of the system is separated into a \emph{bias manager} and a \emph{pressure manager}. Like the engine and transmission module, the managers interact with a set of hardware abstraction interfaces to perform their tasks, and the entire process is overseen by a \emph{module coordinator}.

\begin{figure}[H]
	\centering
	\tikzstyle{big arrow} = [>=latex, line width=4pt, gray]

\begin{tikzpicture}[auto, node distance=2cm, draw=black!70, >=stealth']
  \node [block, minimum width=2cm, inner xsep=0] (stepper) {Stepper Interface};
  \node [block, minimum width=2cm, inner xsep=0, right of=stepper, node distance=3cm] (io) {I/O Interface};

  \node [block, grey shiny, minimum width=5cm, inner xsep=0, below of=stepper, right=-1cm, text width=5cm, node distance=2cm] (bias_manager) {Bias Manager};

  \node [block, minimum width=2cm, inner xsep=0, below of=stepper, node distance=4cm] (can) {CAN Interface};
  \node [block, minimum width=2.2cm, inner xsep=0, below of=io, node distance=4cm] (nv) {NV Storage Interface};
  \node [block, blue shiny, minimum width=2.5cm, text width=2.2cm, inner xsep=0, right of=nv, node distance=3cm] (coordinator) {Module Coordinator};
  \node [block, minimum width=2cm, inner xsep=0, left of=can, node distance=3cm] (adc) {ADC Interface};

  \node [block, grey shiny, minimum width=5cm, inner xsep=0, below of=can, right=-1cm, text width=5cm, node distance=2cm] (pressure_manager) {Pressure Manager};

  \draw [<-, big arrow] (stepper.south) -- ($(bias_manager.north west)!(stepper.south)!(bias_manager.north east)$);
  \draw [->, big arrow] (io.south) -- ($(bias_manager.north west)!(io.south)!(bias_manager.north east)$);

  \draw [<->, big arrow] (can.north) -- ($(bias_manager.south west)!(can.north)!(bias_manager.south east)$);
  \draw [<->, big arrow] (nv.north) -- ($(bias_manager.south west)!(nv.north)!(bias_manager.south east)$);

  \draw [<->, big arrow] (can.south) -- ($(pressure_manager.north west)!(can.south)!(pressure_manager.north east)$);
  \draw [<->, big arrow] (nv.south) -- ($(pressure_manager.north west)!(nv.south)!(pressure_manager.north east)$);

  \draw [<->, big arrow] (bias_manager) -| (coordinator);
  \draw [<->, big arrow] (pressure_manager) -| (coordinator);
  \draw [->, big arrow] (adc) |- (pressure_manager);

  \draw [-, thick] ($(coordinator.south)!0.5!(coordinator.south west)$) |- ($(nv.south)!0.5!(nv.south east)+(0,-0.6cm)$) node [name=x, inner sep=0]{};
  \draw [->, thick] (x) -- ($(nv.south)!0.5!(nv.south east)$);
  \draw [->, thick] (x) -| ($(can.south)!0.5!(can.south east)$);

  \draw [-, thick] (adc.east) -| ($(stepper.south west)+(-0.5cm,-0.6cm)$) node [name=x2, inner sep=0]{};
  \draw [->, thick] (x2) -| ($(stepper.south)!0.5!(stepper.south east)$);
  \draw [->, thick] (x2) -| ($(io.south)!0.5!(io.south east)$);
  
\end{tikzpicture}

	\caption{Block diagram of the braking module software.}
	\label{fig:brake_software_design_block}
\end{figure}

Several hardware abstraction interfaces link the external hardware and micro-controller features to the bias and pressure managers. Unique to the braking module is the \emph{stepper motor interface}. 

\subsubsection{Bias Manager}

The bias manager implements the bias adjustment and bias calibration processes. It listens for requests to initiate either process from the network by way of the CAN interface.

When asked to adjust the brake bias, the bias manager coordinates with the pressure manager to determine if the brakes are being applied, and with the engine and transmission controller to determine if the vehicle is in motion. It performs the calculations required to calculate the new bias position, and adjusts the balance bar accordingly by sending motor step commands to the stepper motor interface.

Bias calibration requires similar coordination to ensure a safe operating state. The bias manager again uses the stepper motor interface to move the balance bar between its extremes. Calibration data is loaded at start-up and stored after calibration in NVRAM by way of the non-volatile storage interface.

\subsubsection{Pressure Manager}

The pressure manager periodically outputs the front and rear brake pressures. It also implements the pressure calibration procedure and listens for calibration requests over the network, similar to the bias manager.

The front and rear pressures are sampled through the ADC interface. The pressure manager uses the CAN interface to output the samples over the network. Current pressure readings are made available to the bias manager to avoid adjusting or calibrating the bias while the brakes are engaged.

The pressure manager communicates with the driver interface through the CAN interface to coordinate the sequence of driver inputs required to calibrate the front and rear brake pressures. Like the pressure manager, calibration data is loaded at start-up and stored after calibration through the non-volatile storage interface.

\subsubsection{Module Coordinator}

Like the module coordinator for the engine and transmission module, the braking module coordinator contains the low-level state machine mechanisms needed by the two managers to transition between their internal states. It keeps track of the current state of the system, and also initializes all of the hardware abstraction interfaces and managers at start-up time. 

\subsubsection{Stepper Motor Interface}

The stepper motor interface takes motor movement commands from the program and converts them into commands for the stepper motor driver. It can direct the stepper motor driver to turn clockwise or counter-clockwise, and move an arbitrary number of steps. 

\subsubsection{GPIO Interface}

The GPIO interface monitors the state of the end-of-travel buttons. It can asynchronously interrupt the program when a change of state occurs. 

\subsubsection{ADC Interface}  

The ADC interface initiates a sampling cycle on either pressure sensor channel. The resulting sample is provided to  the program for further processing. 

\subsubsection{Non-Volatile Storage Interface}

The \emph{non-volatile storage interface} reads and writes calibration data to the non-volatile storage portion of the micro-controller. It is also used to keep track of the current position of the bias bar, so calibration is not required between start-ups.

