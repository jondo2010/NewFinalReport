\begin{abstract}
A Formula SAE vehicle is a performance car built with the primary goal of ranking highly in the dynamic events at the yearly competitions. The goal of this work is to tackle several distinct issues relating 2010 Formula SAE Vehicle constructed at the University of Manitoba. Specifically: to improve electronic control of the transmission, to control the actuator for an adjustable variable-length air intake plenum, to control the brake bias electronically, to broadcast telemetry data from on-board systems to the pit crew, and to create an easy-to-use driver interface and dashboard.

The design and implementation were approached with a divide-and-conquer method. An architecture of four separate electronic modules was developed: an engine and transmission module, a braking module, a telemetry module and a driver interface module. Each module is designed to communicate with the others as well as with pre-existing modules.

Four working prototype modules were constructed with custom-designed printed circuit boards. All hardware was implemented and debugged. All software for the braking module was implemented and tested. Telemetry data was successfully multiplexed and transmitted from the vehicle's two on-board sources to software running on a PC. All low-level driver software for the driver interface and engine/transmission modules was written and tested. A novel electro-pneumatic transmission actuation scheme was developed and modelled and simulated in Simulink. Final testing of the engine/transmission module and the braking module was not conducted because the completed SAE vehicle intended for installation was not yet available.
\end{abstract}

