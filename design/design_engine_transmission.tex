\section{Engine and Transmission Module}


\subsection{Purpose}

Provide intelligent control of the intake runner length and transmission system.

Provide the control signals required to actuate the clutch and gear position levers with the pneumatic system.


\subsection{Processes}


\subsubsection{Variable Intake}


\subsubsection{Gear Selection}


\subsubsection{Advanced Transmission Features}

\begin{description}

\item[Auto Upshift Feature]

This feature of the engine module is aimed primarily at improving performance in the acceleration event. Based on known torque curves, a table of optimal shift points in the RPM range is developed. As the engine reaches the top RPM for a given gear, the engine module will automatically upshift to the next gear, without any driver input. All the driver needs to do is maintain full throttle, and hold on.

\item[Launch]
Full-throttle launch uses an important feature of the ECU called launch control. From a stand-still, the slip ratio of the driven wheels to the non-driven wheels is monitored, and the engine output power is reduced until the ratio reaches 1:1. Drivers use this feature by maintaining full-throttle at the starting line while holding the brake pedal. As soon as the brake pedal is released, the the engine module will release the clutch in a controlled manner in an attempt to get the best possible acceleration.

\item[Crawl]
Part-throttle launch is a feature designed to mimic an automatic transmission. By controlling the clutch position, and thereby modulating the amount of torque transferred to the wheels for a short period of time, the car can be made to creep slowly from a standstill. This will be used when driving up to the starting line of various dynamic events.

\item[Neutral Find]
As drivers come in to the pits from driving the course, a useful feature is the ability for the car to shift the transmission back into neutral to avoid stalling the car. The neutral find feature will automatically downshift the transmission repeatedly until it finds neutral.

\end{description}


\subsection{Software}

Software overview diagram


\subsubsection{Transmission Manager}

Listens to transmission requests from the driver over the network.

Uses the event scheduler to sequence the complex series of actuation vectors required by upshifting and downshifting.

Implement the PID feedback controller used to actuate the pneumatic system.

Interacts with the PWM generator to create the control signals required to actuate the pneumatic system.


\subsubsection{Intake Manager}

Continously monitor engine RPM and throttle position through messages from the ECU.

Will adjust intake runner length based on a functional map of RPM and throttle position.

Use hysterisis to avoid instability.

Map will be generated through dynometer testing.


\subsubsection{Starter Manager}

Listen for driver requests to start the engine.

Provide a means of one-touch starting -- sequences the actual starting of the engine through the solenoid driver. 


\subsubsection{Event Scheduler}

Allow for complex sequencing of events in time.

Controllers may schedule new events to occur at various points in time.

Scheduler will continuously update the schedule and signal the main control loop to execute events that are now current.


\subsubsection{CAN Interface}

Provide a means of interfacing with the physical bus. 

Allow direction of particular messages to particular modules.


\subsubsection{PWM Generator}

Generate PWM signals of at least 20 Hz with a resolution of 1\% duty
cycle.

Two-channel design


\subsubsection{Main Control Loop}

Initializes the system and brings into a known state.

Waits for pending events to be executed and executes them.

Monitor for system faults and react accordingly.


\subsection{Hardware}

Show system hardware diagram


\subsubsection{Microcontroller}

Execute system control software.

In-circuit programmable and debuggable.

Has built-in CAN controller.

Has built-in RAM and ROM as well as EEPROM for holding configuration parameters.


\subsubsection{CAN transceiver}

Interface module with the CAN bus.

Be capable of terminating the bus.


\subsubsection{High current solenoid drivers}

Controlling pneumatic solenoid valves (4)

Controlling starter solenoid


\subsubsection{Bidirectional I/O lines to the ECU}

Bidirectional interface to: launch control, shift cut, traction cut percent, traction control on/off, traction control wet/dry.