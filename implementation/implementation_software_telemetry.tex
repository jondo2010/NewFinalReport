\subsection{Telemetry Module}

The primary objective of the software running on the Telemetry Module is to push data around from various sources to various sinks. It uses the interrupt-driven USART drivers written for the on-board USARTs of the AT90CAN128, as well as the MAX3100 external USART extensively. These USART drivers are discussed in section ?? 

\begin{figure}[H]
\centering
\begin{tikzpicture}[auto, node distance=2cm, draw=black!70, >=stealth']
%  \draw[help lines] (-3,-5) grid (8,2);
  
  % Usart0 section
  \node [blue shiny, rectangle, minimum width=2cm] (usart0) {UART0};
  \node [blue shiny, rectangle, above of=usart0, node distance=1cm] (ecu) {ECU};

  \node [red shiny, rectangle, below of=usart0, right=-1cm, node distance=1cm] (usart0_rx) {Rx};
  \node [red shiny, rectangle, below of=usart0, left=-1cm, node distance=1cm] (usart0_tx) {Tx};

  \draw [<->] (ecu) to (usart0);
  \draw [<-] (usart0_rx) -- ($(usart0.south west)!(usart0_rx.north)!(usart0.south east)$);
  \draw [->] (usart0_tx) -- ($(usart0.south west)!(usart0_tx.north)!(usart0.south east)$);
  
  % Max3100 section
  \node [blue shiny, rectangle, minimum width=2cm, right of=usart0, node distance=3cm] (max3100) {MAX3100};
  \node [blue shiny, rectangle, above of=max3100, node distance=1cm] (xbee) {Xbee};
  \node [blue shiny, rectangle, below of=max3100, above=-0.3cm, left=0.25cm, node distance=0, font=\tiny, inner sep=0.075cm] (max3100_rx_hw) {Rx};

  \node [red shiny, rectangle, below of=max3100, right=-1cm, node distance=1cm] (max3100_rx) {Rx};
  \node [red shiny, rectangle, below of=max3100, left=-1cm, node distance=1cm] (max3100_tx) {Tx};

  \draw [<->] (xbee) to (max3100);
  \draw [<-] (max3100_rx) -- ($(max3100_rx_hw.south west)!(max3100_rx.north)!(max3100_rx_hw.south east)$);
  \draw [->] (max3100_tx) -- ($(max3100.south west)!(max3100_tx.north)!(max3100.south east)$);

  \node [red shiny, rectangle split, rectangle split parts=3, below of=max3100, text width=1.8cm, font=\scriptsize, text centered,
	  minimum width=2cm, anchor=base] (xbee_library)
    {XBee Library
      \nodepart{second} Packetizer
      \nodepart{third} Packet Buffer};

  %\node [red shiny, rectangle, below of=packetizer, node distance=1cm, minimum width=2cm,
%	  text width=1.75cm, font=\scriptsize, text centered] (packet_buf) {Rx Packet Buffer};

  \draw [->] (max3100_rx) -- ($(xbee_library.north west)!(max3100_rx.south)!(xbee_library.north east)$);
  \draw [<-] (max3100_tx) -- ($(xbee_library.north west)!(max3100_tx.south)!(xbee_library.north east)$);

  % Usart1 section
  \node [blue shiny, rectangle, minimum width=2cm, right of=max3100, node distance=3cm] (usart1) {UART1};
  \node [blue shiny, rectangle, above of=usart1, node distance=1cm] (dac) {DAC};

  \node [red shiny, rectangle, below of=usart1, right=-1cm, node distance=1cm] (usart1_rx) {Rx};
  \node [red shiny, rectangle, below of=usart1, left=-1cm, node distance=1cm, dashed] (usart1_tx) {Tx};

  \draw [<->] (dac) to (usart1);
  \draw [<-] (usart1_rx) -- ($(usart1.south west)!(usart1_rx.north)!(usart1.south east)$);
  \draw [->, dashed] (usart1_tx) -- ($(usart1.south west)!(usart1_tx.north)!(usart1.south east)$);

  \node [red shiny, rectangle split, rectangle split parts=4, below of=usart1, text centered, minimum width=2cm, anchor=base, text width=1.85cm, inner xsep=0, font=\scriptsize] (dac_library)
    {DAC Library
      \nodepart{second} Buffer
      \nodepart{third} Decoder
      \nodepart{fourth} Encoder};

  \draw [->] (usart1_rx) -- ($(dac_library.north west)!(usart1_rx.south)!(dac_library.north east)$);
  \draw [<-, dashed] (usart1_tx) -- ($(dac_library.north west)!(usart1_tx.south)!(dac_library.north east)$);

  % CAN stuff
  \node [blue shiny, rectangle, minimum width=2cm, right of=usart1, node distance=3cm] (can) {CAN};
  \node [red shiny, rectangle, minimum width=2cm, below of=can, node distance=1cm] (can_driver) {Driver};

  \node at (dac_library.third split -| can_driver)
	    [red shiny, rectangle, minimum width=2cm, text width=1.85cm, font=\scriptsize, text centered] (can_library) {Telemetry CAN Library};

  \draw [<->] (can) -- (can_driver);
  \draw [<->] (can_driver) -- (can_library);

  \draw [->] (dac_library.third east) -- ($(can_library.north west)!(dac_library.third east)!(can_library.south west)$);
  \draw [<-] (dac_library.fourth east) -- ($(can_library.north west)!(dac_library.fourth east)!(can_library.south west)$);

  % Now for the rest of the connections
  \draw [->] (xbee_library.third west) -| (usart0_tx);
  \draw [<-] (xbee_library.second west) -| (usart0_rx);
  \draw [->] (dac_library.second west) to node [name=mid1] {} (xbee_library.second east);
  \draw [->] (dac_library.fourth west) -| (mid1);

  \node [left of=usart0, node distance=3cm, minimum width=3.5cm] (perifs) {Module Periferals};
  \node [below of=perifs, node distance=1cm, minimum width=3.5cm] (buffers) {Drivers/Buffers};
  \node [below of=buffers, node distance=1cm, minimum width=3.5cm] (libraries) {Libraries};

  \begin{pgfonlayer}{background}
    \path (perifs.north west)+(-0.0,0.1) node (a1) {};
    \path (can.south east)+(+0.1,-0.1) node (b1) {};
    \path[draw=blue!50!black!50, rounded corners=0.1cm] (a1) rectangle (b1);

    \path (buffers.north west)+(-0.0,0.1) node (a2) {};
    \path (can_driver.south east)+(0.1,-0.1) node (b2) {};
    \path [draw=red!50!black!50, rounded corners=0.1cm] (a2) rectangle (b2);

    \path (libraries.north west)+(-0.0,0.1) node (a3) {};
    \path (can_library.south east)+(0.1,-0.1) node (b3) {};
    \path [draw=red!50!black!50, rounded corners=0.1cm] (a3) rectangle (b3);
  \end{pgfonlayer}

  % Guide
  \node at ($(a3 |- b3)+(0.0,-0.1)$) [anchor=north west, blue shiny, minimum width=2cm] (hardware) {Hardware};
  \node [red shiny, right of=hardware, node distance=2.2cm, minimum width=2cm] (software) {Software};
  
\end{tikzpicture}

\caption{Telemetry software overview.}
\label{fig:telemetry_software_implementation}
\end{figure}

\subsubsection{Internal USART Driver}

\begin{figure}[]
\centering
\begin{tikzpicture}[auto, node distance=2cm, draw=black!70, >=stealth']
  \node [start] (int) {Interrupt};
  \node [decision, right of=int, right=0cm] (rx) {Rx Byte?};
  \node [block, below of=rx] (add) {Add to Rx buf.};

  \node [decision, right of=rx, right=0cm] (tx) {Tx Byte?};
  \node [block, below of=tx, text width=2.5cm, inner xsep=0pt] (remove) {Remove from Tx buf.};
  \node [end, right of=tx, right=0cm] (return) {Return};

  \draw [->] (int) -- (rx);
  \draw [->] (rx) to node [] {yes} (add);

  \draw [->] (rx) -- node [name=mid] {no} (tx);
  \draw [->] (add) -| (mid);

  \draw [->] (tx) to node [] {yes} (remove);
  \draw [->] (tx) -- node [name=mid2] {no} (return);
  \draw [->] (remove) -| (mid2);

\end{tikzpicture}
\caption{Basic USART interrupt flow diagram.}
\label{fig:usart_driver_flow}
\end{figure}


\subsubsection{MAX3100 Driver}

An interrupt-driven driver was written to interface with the MAX3100 uart chip which allows asynchronous serial communication with the XBee modem on the Telemetry Module. The same circular buffer approach as with the internal USART drivers was taken, and in fact the drivers use very similar buffering code. Additionally, the software interfaces to the drivers were made as close as possible to allow interchangability by the main module software.

The datahseet for the MAX3100 states that the minimum SPI Clock period is $t_{CP}=238\, ns$, which results in a maximum frequency of approximately $4.2\, MHz$ \cite{MAX3100}. The SPI periferal was thus set to use 1/4 of the system clocks frequency resulting in $4\, MHz$.

The MAX3100 was configured to operate at 115,200 kBaud, which was matched with the XBee modem's interface rate. A simple throughput stress test was performed that continuously wrote data in 16 byte chunks into the driver's software transmit buffer. A fixed-length 32 byte buffer was never overrun with this test, and an actual data throughput of $8.251\, kB/s$ was observed with a logic analyser.

\subsubsection{Xbee Library}

In order to facilitate routing data to and from multiple sources and sinks, it was determined that the Xbee would need to be interfaced using the special "API" mode described in the Xbee documentation \cite{XBeeManual}. The API mode is characterized by a packet interface that needed to be implemented in software. A library was written to implement two critical pieces of functionality:

\begin{enumerate}
\item to set the modem into API mode operation and manage the modem's operating state;
\item and to send and receive packetized data from the modem on behalf of the user software as per the Xbee's API specification \cite{XBeeManual};
\end{enumerate}

The Xbee library implements the binary packet protocol described in the XBee manual, and is able to send and receive unicast and multicast packets. The driver is only capable of sending packets to 16-bit addresses. Full 64-bit address mode was not implemented as the address space was not required in our 3 node network. The driver is also capable of sending command-type packets to the modem and reading the response.

Managing the modem's state was implemented as a state machine. Since minimal functionality was required, only a handful of modem commands were implemented. The driver is able to push the modem into API mode, after which all communication is done using the packet interface.

The modem commands implemented are listed in Table \ref{tab:xbee_commands}.

\begin{table}
\caption{Implemented Xbee modem commands.\label{tab:xbee_commands}}
\centering{}
\begin{tabular}{|l|l|}
\hline 
Command & Description \tabularnewline
\hline
\hline
AP & Set and read the API mode state \tabularnewline
\hline
CH & Set and read the RF channel \tabularnewline
\hline 
ID & Set and read the PAN (Personal Area Network) ID \tabularnewline
\hline
MY & Set and read the modems local address \tabularnewline
\hline
\end{tabular}
\end{table}

Similar to the rest of the software drivers implemented, the Xbee library uses callbacks to provide the user software with incoming data. Function pointers are used to connect the library to a UART, which makes it possible to reconfigure the library to use a different USART on-the-fly.

\subsubsection{DAC Library}

To reduce the implementation work required by us, we asked David Schilling, a computer science student on the Formula SAE team, to write the DAC software library to read and write the DAC's serial format, given the requirements described in Section \ref{sec:Telemetry-Module-Design}.

The DAC library maintains it's own buffer of incoming DAC data. We also asked Dave to provide seperate functions for buffering incoming data, and for processing that data. This seperation allowed the library to be easily tied in with the rest of the system software on the telemetry module: adding a data to the DAC buffer occurs whenever the internal USART driver interrupts with a new byte, and processing is called from the mainline loop.

\subsubsection{Main Control Loop}