\section{Braking Module Design\label{sec:Braking-Module-Design}}


\subsection{Purpose}

Adjust the brake bias position electronically and on-demand from the driver.

Provide calibration sequences to allow for mechanical wear.

The member of the Formula SAE team responsible for the mechanical design of the brake pedal box and bias system was not able to meet our time constraints in providing us with the torque requirements for the bias bar actuation. It was therefore decided at design-time that the implementation should be flexible with regards to which stepper motor will actually be chosen.


\subsection{Processes}


\subsubsection{Brake Bias Calibration}


\subsubsection{Brake Bias Adjustment}


\subsubsection{Pressure Sensor Calibration}


\subsection{Software}

Software overview diagram


\subsubsection{Bias Manager}

Handles requests to adjust the brake bias from the network. 

Performs balance bar calibration sequence on demand.


\subsubsection{Pressure Manager}

Continuously samples brake pressure.

Performs a pressure calibration sequence on demand.


\subsubsection{CAN Interface}

As documented above.


\subsubsection{Main Control Loop}

Initializes the system and brings into a known state.

Monitor for system faults and react accordingly.


\subsection{Hardware}


\subsubsection{Microcontroller}

As mentioned above.


\subsubsection{CAN transceiver}

As mentioned above.


\subsubsection{Stepper Motor Driver}

Generates signals required to actuate the stepper motor used to adjust
the balance bar.


\subsubsection{Analogue-to-Digital Converter}

Samples both front and rear pressure sensors. Capable of 10-bit resolution.


\subsubsection{End-of-Travel Sensing Switches}