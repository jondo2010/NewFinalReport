\section{Transmission}

\subsection{Overview}

The Formula SAE 2010 vehicle uses a 6-speed manual sequential transmission to transmit power from the engine to the drive train. As in other types of manual transmissions, the sequential transmission works by engaging and disengaging several sets of gears with shift forks to obtain different gear ratios. A cast steel drum inside the transmission, called the \emph{shift drum}, has a series of complex grooves machined into it, in which the shift forks ride. Rotating the drum causes the forks to engage one gear and disengage the other. A ratcheting mechanism allows a single external lever, called the \emph{shift lever}, to rotate the drum forward and back by a discrete amount. Each full movement of the lever forwards or backwards causes a gear change up or down, respectively \cite{HowtoManualTransmission, cbr600}.

\subsection{Clutch}

The CBR600f4i is equipped with a multi-plate clutch which serves to transmit torque from the engine to the drivetrain. Layers of friction plates in the clutch are forced to rotate together when the clutch is engaged by a series of pre-tensioned springs. When the clutch is disengaged externally via the clutch lever, the plates are seperated and the drivetrain is allowed to spin freely from the engine. The clutch lever is actuated on the stock motorcycle with a hand lever.

A major state of interest of the clutch is the transition from fully disengaged to fully engaged, and vice-versa. In this state the clutch plates slip as one rotates faster than the other. \Citet{clutch_control} describe the torque transmitted through the clutch in slipping, $T_c{slip}$, as

\begin{equation}\label{eq:clutch_slip}
  T_c{slip}=F_n\mu R_a \cdot sgn\left(\omega_e-\omega_c\right)
\end{equation}

where $F_n$ is the normal force on the clutch plates, $\mu$ the coefficient of friction on the plate surfaces, and $R_a$ the radius of the plates, and $\omega_e$ and $\omega_c$ are the rotational velocities of the engine, and clutch discs, respectively. This shows that the amount of torque transmitted depends dynamically on the normal force $F_n$, which is proportional to the pressure holding the plates together.

In normal operation of the car, the clutch is used in two specific cases:
\begin{enumerate}
  \item Accelerating the car from a stand-still, and
  \item Downshifting the transmission\footnote{Use of the clutch is not required for upshifting, only a small reduction in engine torque.}.
\end{enumerate}

When accelerating the car from a stand-still, two further cases need to be considered: accelerating at the beginning of a race, or \emph{launching}, and moving the car forward in a slow, controlled, manner, which we have called \emph{crawling}. In launching the car, the goal is to gain as much momentum as possible by transmitting as much torque from the engine to the wheels as possible without breaking the tires loose. The goal of crawling the car is to accelerate in a slow, completely controlled manner.

With purely manual clutch control (i.e., with a clutch lever), skillful completion both of these operations requires a significant amount of skill on the part of the driver. To launch the car with minimal tire slip, the driver must modulate the throttle as well as the clutch position (allowing a certain amount of slip). Crawling a Formula SAE car is a similar operation, however due to the low torque output at low RPM of the engine, it is very easy to stall the engine by engaging the clutch too quickly. Additionally, it is often desirable to move the car slower than would be obtained by driving in first gear with the clutch fully engaged. This must be done by never fully engaging the clutch: the clutch state is alternated between slipping and fully disengaged.

Downshifting the transmission (changing from a higher gear to a lower gear) requires that the clutch be fully disengaged. In the fully manual operation, the driver must disengage the clutch, shift the transmission, blip the throttle\footnote{``Blipping'' the throttle is a short increase in throtttle used to increase the engine RPM in order to match it closer to the clutch plate RPM}, and then re-engage the clutch. This is required since the torque transmitted by the clutch (Eq. \ref{eq:clutch_slip}) can be in the reverse direction, from the wheels to the engine.

\subsection{Gear Selector}

The gear selector is actuated by a lever attached to the engine body. The gear selector may be rotated in either direction from its rest point. Up-shifting is accomplished by rotating the lever in one direction, while down-shifting is accomplished by rotating the lever in the opposite direction. 

\subsection{Neutral Sensing}

A \emph{neutral sense} wire is attached to the shift drum. The wire is grounded when the transmission is in neutral. The wire is left floating when the vehicle is in gear. 

\subsection{Gear Position Sensing}

The stock transmission has no means of determining what gear the transmission is in; it can only determine when the transmission is in neutral using the aforementioned neutral sense wire. The ECU has provisions for sensing gear position by reading a geared potentiometer that can be attached to the shift drum, however this requires mechanical modification of the shift drum itself.

\subsection{Previous Implementations and Shortcomings}

There are two major downfalls to operating a purely manual transmission: it requires a great deal of dexterity and effort on behalf of the driver, and shifting time is relatively slow (around a second) compared to an electronically-assisted transmission (under one-hundred milliseconds.) 

For these reasons, the Formula SAE vehicle switched to an electronically-assisted transmission system in the 2009 competition year. This implementation replaced a purely mechanical clutch pedal and gear shifter with electropneumatic actuators linked directly to the clutch and gear selector levers. Driver control was achieved with a set of paddles located on the steering wheel: an up-shift paddle and a down-shift paddle.

The electropneumatic system relieved the driver from the burden of manually engaging the clutch. It also reduced shift-time by a factor of nearly ten. However, the system made use of binary pneumatic valves which reduced the ability to engage the clutch smoothly. The rate at which the clutch was engaged was limited by the air flow rate of the valve itself, resulting in a rough shifting sequence that was mechanically hard on the transmission. 

Another limitation was posed by the lack of gear position sensing in the stock transmission. Although the system was able to determine when the transmission was in neutral, there was no way to determine if the gears had meshed after up- or down-shifting. To do so would require modifying the shift drum to incorperate a gear-sensing potentiometer. This led to situations where the transmission controller became confused as to which gear was actually selected, and required driver intervention to resynchronize the controller with the transmission.

