
\chapter{CAN Diagnostic Tool Input Format}
\label{apx:candt}

The CAN diagnostic tool can be programmed to output a sequence of messages onto the bus at pre-determined times. As mentioned in Sec.\ \ref{sec:implementation_candt}, the packet sequence is input via a serial link. This appendix describes the format of this input.

\nomenclature{ASCII}{American Standard Code for Information Interchange}
\section*{Input Format}

Each packet is input on a single line that is terminated with the new-line character (ASCII Character 10). The fields are colon-separated. Each line contains the following fields:

\begin{itemize}
	\item the \emph{packet type} field, which specifies a data or remote request packet;
	\item the \emph{injection time} field, which indicates the time at which the packet should be broadcast;
	\item the \emph{identifier mode} field, which specifies if the packet uses the extended identifier feature;
	\item the \emph{identifier} field, which defines the ID with which the packet will be broadcast;
	\item the \emph{data length} field, which indicates how many bytes will be broadcast; and
	\item the \emph{data} field, which includes the actual data bytes to be broadcast.
\end{itemize}

\subsection*{Packet Type}

Packet type can be either a data packet or remote request packet. A data packet is specified by the `@' character, while a remote request packet is specified by the `\$' character.

\subsection*{Injection Time}

Injection time is specified in milliseconds from the start of the injection sequence, and must be a 32-bit unsigned integer value in decimal form. Packets must be input in ascending injection time.

\subsection*{Identifier Mode}

The identifier mode can be set to either standard or extended. A standard identifier type is specified by a `0' character, while an extended identifier type is specified by a `1' character.

\subsection*{Identifier}

The identifier can be either an 11-bit value (for the standard identifier type) or a 29-bit value (for the extended identifier type). The field itself expects a 32-bit unsigned integer in hexadecimal format with a leading {}``0x''.

\subsection*{Data Length}

Both data packets and remote request packets can include data. CAN packets are limited packets to 8 bytes of data. This field is an 8-bit unsigned integer value in decimal form, ranging from 0 to 8.

\subsection*{Data}

Bytes to be broadcast are specified as pairs of ascii characters representing the hexadecimal equivalent of the data byte, e.g., the corresponding data field for $1A2B3C4D_{16}$ would be {}``1A2B3C4D''.

\subsection*{Input Termination}

Input is terminated when a single exclamation mark (`!') followed by a new-line character is transmitted. 

\section*{Example}

In this example, we broadcast four packets, outlined in Table \ref{table:candt_packet_example}.

\begin{table}[H]
\caption{List of packets to be broadcast for the input format example.}
\centering
\begin{tabular}{|c|c|c|c|c|c|}
	\hline 
	Packet Type & Injection Time (ms) & Identifier Mode & Identifier &	 Data \\
	\hline
	\hline
	Data & 1000 & Standard & 0x7FF & 0x1A 0x2B 0x3C \\
	\hline
	Remote & 2000 & Standard & 0x1AB & \emph{no data} \\
	\hline
	Remote & 2500 & Extended & 0x1FFFFF & 0x01 \\
	\hline
	Data & 4000 & Extended & 0x1AC90 & 0x00 0x01 0x02 \\
	\hline
\end{tabular}
\label{table:candt_packet_example}
\end{table}

A listing of the corresponding input is shown in Fig.\ \ref{fig:candt_packet_example}.

\begin{figure}[H]
	\centering
	\makebox[\textwidth]{\hrulefill}
	\begin{verbatim}
	@:1000:0:0x7FF:3:1A2B3C
	$:2000:0:0x1AB
	$:2500:1:0x1FFFFF:1:01
	@:4000:1:0x1AC90:3:000102
	!
	\end{verbatim}
	\makebox[\textwidth]{\hrulefill}
	\caption{Listing of the corresponding input for the input format example.}
	\label{fig:candt_packet_example}
\end{figure}
