\section{Braking Module Design\label{sec:Braking-Module-Design}}

\subsection{Overview}

The braking module adjusts the brake bias electronically when requested by the driver. It also provides the ability to calibrate itself to account for mechanical wear and a degree of component tolerance that may cause unwanted rotation of the balance bar. 

Figure \ref{fig:design_brake_overview_block} shows an overview of the braking module and it's interactions with the environment. 

\begin{figure}[H]
\centering
\begin{tikzpicture}[auto, node distance=2cm, draw=black!70, >=stealth', font=\footnotesize]
  \node [block, minimum width=2cm, inner xsep=0] (pressure1) {Front Pressure Sensor};
  \node [block, minimum width=2cm, inner xsep=0, right of=pressure1, node distance=4cm] (pressure2) {Front Pressure Sensor};

  \node [block, blue shiny, minimum width=6cm, text width=4cm, below of=pressure1, right=-1cm, node distance=1.5cm] (module) {Braking Module};

  \node [block, below of=pressure1, right=-1cm, node distance=3cm, text width=1.1cm] (eot1) {Left EOT Switch};
  \node [block, below of=pressure2, left=-1cm, node distance=3cm, text width=1.1cm] (eot2) {Right EOT Switch};

  \node at ($(eot1)!0.5!(eot2)$) [red shiny, circle, label={[text width=1.5cm, rotate=90, below=5pt, left=10pt]below right:Stepper Motor}] (motor) {M};

  \node [rectangle, below of=motor, minimum width=2cm, label=below:Bias Bar, pattern=north east lines, node distance=1cm] (bias) {};

  \draw [->, thick] (pressure1.south) to ($(module.north west)!(pressure1.south)!(module.north east)$);
  \draw [->, thick] (pressure2.south) to ($(module.north west)!(pressure2.south)!(module.north east)$);

  \draw [->, thick] (eot1.north) to ($(module.south west)!(eot1.north)!(module.south east)$);
  \draw [->, thick] (eot2.north) to ($(module.south west)!(eot2.north)!(module.south east)$);

  \draw [<-, thick] (motor.north) to ($(module.south west)!(motor.north)!(module.south east)$);
  \draw [->, thick, dashed] (motor.south) to ($(bias.north west)!(motor.south)!(bias.north east)$);

  \draw [->, thick, dashed] (bias) -| (eot1);
  \draw [->, thick, dashed] (bias) -| (eot2);

  %%% CAN Bus

  \node [bus, name=can1, right of=module, label={[rotate=90, left=0.75cm]below right:CAN Bus}, node distance=3.5cm] {CAN Bus};
  \node [bus, name=can2, below of=can1, node distance=1cm] {};
  \node [bus, name=can3, above of=can1, node distance=1cm] {};

  \draw [<->, thick] (module) to (can1);
  \draw [-, line width=3pt] (can1) -- (can2);
  \draw [-, line width=3pt] (can1) -- (can3);

  %%% Legend

  \draw [->, thick] ($(eot2.east)+(2cm,0cm)$) -- ++(0.5cm,0) node [label={[font=\tiny]below:Electrical}] {} -- ++(0.5cm,0);
  \draw [->, thick, dashed] ($(eot2.south east)+(2cm,0cm)$) -- ++(0.5cm,0) node [label={[font=\tiny]below:Mechanical}] {} -- ++ (0.5cm,0);
\end{tikzpicture}
\caption{Braking module overview block diagram.}
\label{fig:design_brake_overview_block}
\end{figure}

A \emph{stepper motor} is connected to the balance bar and used to rotate it to the desired position. The motor must be bi-directional and have enough steps per revolution to provide the resolution required to allow a 0.5\% bias ratio adjustment.

Two \emph{end-of-travel switches} are used to determine when the balance bar is at it's end of travel. These are required to calibrate the system. 

\emph{Pressure sensors} on the front and rear braking cylinders are used to determine the current pressure in each system. This is critical for safety, as we cannot rotate the balance bar when the brakes are engaged. As noted in \ref{sec:background_braking}, pressure in the braking systems can reach upwards of \unit{7000}{\kilo\pascal}. The pressure sensors chosen for the implementation must be capable of reading such pressures.

\subsection{Processes}

\subsubsection{Brake Bias Calibration}

A flow-chart of the brake bias calibration procedure is shown in Fig. \ref{fig:brake_bias_calibration_flow}. By recording the number of steps it takes to rotate the balance bar from it's leftmost extreme to it's rightmost extreme, we can determine the correct centre point for the bias bar. 

\begin{figure}[H]
\centering
\begin{tikzpicture}[auto, node distance=2cm, draw=black!70, >=stealth', font=\scriptsize]
  \node [start] (start) {Start};

  \node [block, right of=start, text width=1cm, inner xsep=0pt] (step_left) {Step Left};
  \node [decision, right of=step_left, text width=1.0cm, inner sep=0pt] (at_left) {At leftend?};

  \node [block, right of=at_left, text width=1cm, inner xsep=0pt] (step_right) {Step Right};
  \node [block, right of=step_right, text width=1cm, inner xsep=2pt] (count) {Add to count};
  \node [decision, right of=count, text width=1cm, inner sep=0pt] (at_right) {At rightend?};

  \node [block, below of=at_right, text width=2cm, inner xsep=2pt, node distance=2.5cm] (adjust) {Adjust bias to old value};
  \node [block, left of=adjust, text width=1.5cm, inner xsep=2pt, node distance=3cm] (store) {Store calibration};
  \node [end, left of=store, node distance=7cm] (end) {Done};

  \draw [->, thick] (start) to node[coordinate, name=x1]{} (step_left);
  \draw [->, thick] (step_left) to (at_left);
  \draw [->, thick] (at_left.south) -- ++(0,-0.25) node[below]{no} -| (x1);
  \draw [->, thick] (at_left.east) to node[name=x2]{yes} (step_right);
  \draw [->, thick] (step_right) to (count);
  \draw [->, thick] (count) to (at_right);
  \draw [->, thick] (at_right.south) -- ++(0,-0.25) node[below]{no} -| (x2);

  \draw [->, thick] (at_right.east) -- ++(0.5,0) node[above]{yes} |- (adjust);
  \draw [->, thick] (adjust) to (store);
  \draw [->, thick] (store) to (end);
\end{tikzpicture}

\caption{Brake bias calibration sequence.}
\label{fig:brake_bias_calibration_flow}
\end{figure}

The first step in calibrating the brake bias is to rotate the balance bar to it's leftmost point of travel. when the bar reaches this point, it pushes on the left end-of-travel switch, which signals the module that the balance bar is at it's leftmost extreme. 

The next step is to rotate the balance bar to it's rightmost point of travel. The module counts each step taken until the balance bar reaches it's rightmost extreme and pushes on the right end-of-travel switch.

At this point, the module is now aware of how many steps are required to move from left to right and thus can re-centre the bar. The module saves the number of steps required to travel between the two extremes in some non-volatile parameter storage to be specified later, and adjusts the bias ratio to the last set point before calibration.

\subsubsection{Brake Bias Adjustment}

The brake bias adjustment procedure is shown in Fig. \ref{fig:design-braking-bias-adjustment}. 

When a request to adjust the bias is relayed to the module, the desired bias ratio is validated to ensure an adjustment that is out of range of the balance bar is not attempted. 

Once validated, the module calculates the number of steps required to move from the current position to the desired offset. It then signals the motor to advance this many steps in the correct direction.

\subsubsection{Pressure Sensor Calibration}

\begin{figure}
\centering
\begin{tikzpicture}[auto, node distance=2cm, draw=black!70, >=stealth', font=\scriptsize]
  \node [start] (start) {Start};
  \node [decision, right of=start, right=0cm, text width=1.4cm, inner sep=0pt] (min) {Min Applied?};
  \node [block, right of=min, text width=2cm, inner xsep=0pt, node distance=3cm] (sample1) {Sample Pressures};
  \node [decision, right of=sample1, right=0cm, text width=1.4cm, inner sep=0pt] (max) {Max Applied?};
  \node [block, right of=max, text width=2cm, inner xsep=0pt, node distance=3cm] (sample2) {Sample Pressures};

  \node [decision, below of=sample2, text width=1.4cm, inner sep=0pt, node distance=2.5cm] (valid) {Valid samples?};
  \node [block, left of=valid, text width=1.5cm, inner xsep=2pt, node distance=3cm] (save) {Save Calibration};

  \node [end, below of=save] (success) {Success};
  \node [end, below of=valid] (fail) {Fail};

  \draw [->, thick] (start) -- ($(start.east)!0.5!(min.west)$) node[name=x1,coordinate]{} -- (min);
  \draw [->, thick] (min.south) -- ++(0,-0.25cm) node[below]{no} -| (x1);
  \draw [->, thick] (min) to node [] {yes} (sample1);

  \draw [->, thick] (sample1) to node [name=x2]{} (max);
  \draw [->, thick] (max.south) -- ++(0,-0.25cm) node[below]{no} -| (x2);
  \draw [->, thick] (max) to node [] {yes} (sample2);

  \draw [->, thick] (sample2) to (valid);
  \draw [->, thick] (valid) to node [above] {yes} (save);

  \draw [->, thick] (save) to (success);
  \draw [->, thick] (valid.south) to node [] {no} (fail);

\end{tikzpicture}
\caption{Brake pressure calibration sequence.}
\label{fig:design_brake_pressure_calibration_flow}
\end{figure}

\subsection{Software}

\begin{figure}
\centering
\tikzstyle{big arrow} = [>=latex, line width=4pt, gray]

\begin{tikzpicture}[auto, node distance=2cm, draw=black!70, >=stealth']
  \node [block, minimum width=2cm, inner xsep=0] (stepper) {Stepper Interface};
  \node [block, minimum width=2cm, inner xsep=0, right of=stepper, node distance=3cm] (io) {I/O Interface};

  \node [block, grey shiny, minimum width=5cm, inner xsep=0, below of=stepper, right=-1cm, text width=5cm, node distance=2cm] (bias_manager) {Bias Manager};

  \node [block, minimum width=2cm, inner xsep=0, below of=stepper, node distance=4cm] (can) {CAN Interface};
  \node [block, minimum width=2.2cm, inner xsep=0, below of=io, node distance=4cm] (nv) {NV Storage Interface};
  \node [block, blue shiny, minimum width=2.5cm, text width=2.2cm, inner xsep=0, right of=nv, node distance=3cm] (coordinator) {Module Coordinator};
  \node [block, minimum width=2cm, inner xsep=0, left of=can, node distance=3cm] (adc) {ADC Interface};

  \node [block, grey shiny, minimum width=5cm, inner xsep=0, below of=can, right=-1cm, text width=5cm, node distance=2cm] (pressure_manager) {Pressure Manager};

  \draw [<-, big arrow] (stepper.south) -- ($(bias_manager.north west)!(stepper.south)!(bias_manager.north east)$);
  \draw [->, big arrow] (io.south) -- ($(bias_manager.north west)!(io.south)!(bias_manager.north east)$);

  \draw [<->, big arrow] (can.north) -- ($(bias_manager.south west)!(can.north)!(bias_manager.south east)$);
  \draw [<->, big arrow] (nv.north) -- ($(bias_manager.south west)!(nv.north)!(bias_manager.south east)$);

  \draw [<->, big arrow] (can.south) -- ($(pressure_manager.north west)!(can.south)!(pressure_manager.north east)$);
  \draw [<->, big arrow] (nv.south) -- ($(pressure_manager.north west)!(nv.south)!(pressure_manager.north east)$);

  \draw [<->, big arrow] (bias_manager) -| (coordinator);
  \draw [<->, big arrow] (pressure_manager) -| (coordinator);
  \draw [->, big arrow] (adc) |- (pressure_manager);

  \draw [-, thick] ($(coordinator.south)!0.5!(coordinator.south west)$) |- ($(nv.south)!0.5!(nv.south east)+(0,-0.6cm)$) node [name=x, inner sep=0]{};
  \draw [->, thick] (x) -- ($(nv.south)!0.5!(nv.south east)$);
  \draw [->, thick] (x) -| ($(can.south)!0.5!(can.south east)$);

  \draw [-, thick] (adc.east) -| ($(stepper.south west)+(-0.5cm,-0.6cm)$) node [name=x2, inner sep=0]{};
  \draw [->, thick] (x2) -| ($(stepper.south)!0.5!(stepper.south east)$);
  \draw [->, thick] (x2) -| ($(io.south)!0.5!(io.south east)$);
  
\end{tikzpicture}

\caption{Brake module software block diagram.}
\label{fig:pneumatics_design}
	\end{figure}

\subsubsection{Bias Manager}

Handles requests to adjust the brake bias from the network. 

Performs balance bar calibration sequence on demand.


\subsubsection{Pressure Manager}

Continuously samples brake pressure.

Performs a pressure calibration sequence on demand.


\subsubsection{CAN Interface}

As documented above.


\subsubsection{Main Control Loop}

Initializes the system and brings into a known state.

Monitor for system faults and react accordingly.


\subsection{Hardware}

\begin{figure}
\centering
\begin{tikzpicture}[auto, node distance=2cm, draw=black!70, >=stealth']
%  \draw[help lines] (-3,-5) grid (8,2);

  %%% Pressure Sensors
  \node [block, name=pressure1, font=\scriptsize, text width=1.5cm, inner xsep=0] {Front Pressure Sensor};
  \node [block, name=pressure2, right of=pressure1, font=\scriptsize, text width=1.5cm, inner xsep=0] {Rear Pressure Sensor};

  %%% Balance bar and EOT switches
  \node [block, name=eot1, right of=pressure2, font=\scriptsize, text width=1.5cm, node distance=3cm] {Left EOT Switch};
  \node [block, name=balance, right of=eot1, font=\scriptsize, text width=1.25cm, inner xsep=0] {Balance Bar};
  \node [block, name=eot2, right of=balance, font=\scriptsize, text width=1.5cm] {Right EOT Switch};

  \node [red shiny, circle, below of=balance, name=motor, node distance=1cm, inner sep=1pt] {M};
  \node [block, below of=motor, node distance=1cm, text width=1.5cm, font=\scriptsize] (driver) {Stepper Driver};

  \draw [->, dashed, thick] (balance) -- (eot1);
  \draw [->, dashed, thick] (balance) -- (eot2);
  \draw [->, dashed, thick] (motor) -- (balance);
  \draw [->, thick] (driver) -- (motor);

  \node [block, name=adc1, below of=pressure1, text width=1.5cm] {ADC};
  \node [block, name=adc2, below of=pressure2, text width=1.5cm] {ADC};

  %%% Microcontroller block
  \path ($(adc1.south west)+(0,-1cm)$) node (mu1) {};
  \path ($(adc1.south west -| eot2.south east) + (0,-1cm)$) node (mu2) {};

  \node [block, name=micro, fit=(mu1) (mu2), inner xsep=0, minimum height=1cm] {Microcontroller};

  \draw [->, thick] (pressure1) to (adc1);
  \draw [->, thick] (pressure2) to (adc2);
  \draw [->, thick] (adc1.south) -- ($(micro.north west)!(adc1.south)!(micro.north east)$);
  \draw [->, thick] (adc2.south) -- ($(micro.north west)!(adc2.south)!(micro.north east)$);

  \draw [->, thick] (eot1.south) -- ($(micro.north west)!(eot1.south)!(micro.north east)$);
  \draw [->, thick] (eot2.south) -- ($(micro.north west)!(eot2.south)!(micro.north east)$);

  \draw [<-, thick] (driver.south) -- ($(micro.north west)!(driver.south)!(micro.north east)$);

  %%% CAN Bus

  \node at ($(micro.east)+(2cm,0)$) [block, name=can, node distance=2cm, inner xsep=2pt] {CAN Transceiver};

  \draw [<->, thick] (micro) to (can);

  \node [bus, name=can1, above of=can, label=above:CAN Bus, node distance=2.5cm] {CAN Bus};
  \node [bus, name=can2, left of=can1, node distance=1cm] {};
  \node [bus, name=can3, right of=can1, node distance=1cm] {};

  \draw [-, line width=3pt] (can) -- (can1);
  \draw [-, line width=3pt] (can1) -- (can2);
  \draw [-, line width=3pt] (can1) -- (can3);

  %%% Background

  \begin{pgfonlayer}{background}
    \path (adc1.north west)+(-0.3,0.3) node (a) {};
    \path (can.south east)+(+0.2,-0.2) node (b) {};
    \path[module] (a) rectangle (b);
  \end{pgfonlayer}

  %%% Legend

  \draw [->, thick] ($(micro.south west)+(0.5cm,-0.5cm)$) -- ++(0.5cm,0) node [label={[font=\tiny]below:Electrical}] {} -- ++(0.5cm,0);
  \draw [->, thick, dashed] ($(micro.south west)+(2cm,-0.5cm)$) -- ++(0.5cm,0) node [label={[font=\tiny]below:Mechanical}] {} -- ++ (0.5cm,0);
\end{tikzpicture}

\caption{Brake module hardware block diagram.}
\label{fig:pneumatics_design}
\end{figure}


\subsubsection{Microcontroller}

As mentioned above.


\subsubsection{CAN transceiver}

As mentioned above.


\subsubsection{Stepper Motor Driver}

Generates signals required to actuate the stepper motor used to adjust
the balance bar.


\subsubsection{Analogue-to-Digital Converter}

Samples both front and rear pressure sensors. Capable of 10-bit resolution.


\subsubsection{End-of-Travel Sensing Switches}