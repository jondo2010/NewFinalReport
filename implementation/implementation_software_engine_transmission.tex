\subsection{Engine and Transmission Module}

<Software interface map>


\subsubsection{Transmission Manager}


\subsubsection{Intake Manager}


\subsubsection{Starter Manager}


\subsubsection{Event Scheduler}


\subsubsection{CAN Interface}

<Data flow diagram>


\subsubsection{PWM Generator}

An efficient method was devised to generate 2 synchronized PWM signals from the L9822E driver chip. The $32\, kHz$ external crystal was used as the input to the 8-bit TIM2 timer periferal on the microcontroller. The input to the timer was first scaled by 8 which provided a timebase:
\begin{equation}
TB=\frac{32.768\, kHz}{8}=4.096\, kHz
\end{equation}

The timebase period is then given by:
\begin{equation}
\frac{1}{TB}\approx244\,\mu{S}
\end{equation}

We then define a constant scaling factor for the PWM generator:
\begin{equation}
{PWM\_DUTY\_SCALE}=\frac{T_{PWM}}{TB}\approx205
\end{equation}

By loading the timers compare register with with a value scaled with the constant scaling factor, we can cause the timer to generate an interrupt precisely when a level change in the PWM is required.

When generating 2 waveforms with the same timer periferal, 8 combinations of duty cycles of channels A and B were identified, and can be seen in Figure \ref{fig:pwm_cases}.

Since the waveforms are synchronized, it can be seen that there are only 2 cases where 3 transitions per period are required, corresponding to \ref{fig:pwm_cases_1} and \ref{fig:pwm_cases_2}. This happens when both channels have $0<Duty<100\%$. 2 cases are also apparent when no transitions are required, shown in \ref{fig:pwm_cases_7} and \ref{fig:pwm_cases_8}.

An efficient generating routine was constructed to effect the level transitions and to reset the timer to interrupt at the next transition. For the two cases requiring 3 transitions per period, the timer interrupts 3 times per period. For the two cases where both channels have duty cycle between 0 and 100\%, the routine still interrupts once to allow for a change in duty cycle. For the rest of the cases described, the timer only interrupts twice per PWM period.

\begin{figure}[ht]
  \centering
  \subfigure[Case 1]
  {
    \label{fig:pwm_cases_1}
    \begin{tikztimingtable}
      $PWM_A$		& G 4H 4L \\
      $PWM_B$		& G 2H 6L \\
    \end{tikztimingtable}
  }
  \subfigure[Case 2]
  {
  \label{fig:pwm_cases_2}
  \begin{tikztimingtable}
    $PWM_A$		& G 4H 4L \\
    $PWM_B$		& G 6H 2L \\
  \end{tikztimingtable}
  }
  \subfigure[Case 3]
  {
  \label{fig:pwm_cases_3}
  \begin{tikztimingtable}
    $PWM_A$		& G 4H 4L \\
    $PWM_B$		& 8L \\
  \end{tikztimingtable}
  }
  \subfigure[Case 4]
  {
  \label{fig:pwm_cases_4}
  \begin{tikztimingtable}
    $PWM_A$		& G 4H 4L \\
    $PWM_B$		& G 8H G \\
  \end{tikztimingtable}
  }
  \subfigure[Case 5]
  {
  \label{fig:pwm_cases_5}
  \begin{tikztimingtable}
    $PWM_A$		& G 8H G \\
    $PWM_B$		& G 4H 4L \\
  \end{tikztimingtable}
  }
  \subfigure[Case 6]
  {
  \label{fig:pwm_cases_6}
  \begin{tikztimingtable}
    $PWM_A$		& 8L \\
    $PWM_B$		& G 4H 4L \\
  \end{tikztimingtable}
  }
  \subfigure[Case 7]
  {
  \label{fig:pwm_cases_7}
  \begin{tikztimingtable}
    $PWM_A$		& 8L \\
    $PWM_B$		& 8L \\
  \end{tikztimingtable}
  }
  \subfigure[Case 8]
  {
  \label{fig:pwm_cases_8}
  \begin{tikztimingtable}
    $PWM_A$		& G 8H G \\
    $PWM_B$		& G 8H G \\
  \end{tikztimingtable}
  }
  \caption{PWM Cases (1 period shown).}
  \label{fig:pwm_cases}
\end{figure}

\subsubsection{Main Control Loop}