\section{Driver Interface}

The driver interface targets several major goals: providing access to critical features with a minimal amount of effort, providing visual feedback of the state of the vehicle without distracting the driver, and allowing the driver and pit crew to adjust the vehicle dynamic parameters as quickly and easily as possible. This is accomplished with the introduction of a driver display integrated into the steering wheel, and the ability to choose a set of vehicle parameters on the fly.

\subsection{Driver Controls}

Changing gears and accessing the automatic features of the transmission should be easy for the driver and require minimal effort. The driver also needs a means of actually starting the motor.

\begin{itemize}
\item Provide a means of shifting gears without manual timing or clutching.
\item Allow the driver to enable or disable all of the transmission features such as the auto up-shift feature, etc.
\item Provide the driver with a button to start or stop the engine. 
\end{itemize}

\subsection{Diagnostic Information}

The driver can maximize their performance when they are fully aware of the vehicle's state, and they can also avoid engine damage by knowing when it is reaching its operating limits. Diagnostic information can provide this to the driver, but it must be presented in a way that avoids distracting or overburdening the driver. Only critical features should have prominence on the display.

\begin{itemize}
\item Critical warnings regarding oil pressure and engine temperature must be delivered to the driver on an easy-to-read display.
\item The current gear and RPM should be displayed prominently on the screen so that they are readable at all times.
\item Supplementary vehicle performance information such as fuel level, oil pressure, and battery voltage should be made available to the driver without being a distraction.
\item The wireless telemetry link status should be displayed at all times.
\end{itemize}

\subsection{Vehicle Dynamic Adjustment}

Allowing the driver to adjust the electronically tuneable parameters of the vehicle on the fly means the driver can optimize performance of the vehicle for particular events at competition. Incorporating the ability to save the settings allows different drivers to setup the vehicle to their liking and dial-up the settings whenever they get behind the wheel.

\begin{itemize}
\item Provide an easy to use interface graphical interface to adjust vehicle dynamic parameters.
\item Allow the driver to choose between sets of dynamic vehicle parameters or `operating modes.'
\item Allow an operating mode to be selected with a single knob.
\item Disable adjustment of vehicle parameters above speeds of \unit{25}{\kilo\metre\per\hour}.
\item Allow the tuning of individual dynamic parameters and overwriting of preset operating modes.
\end{itemize}
