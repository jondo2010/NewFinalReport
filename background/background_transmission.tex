\section{Transmission}

\subsection{Overview}

The Formula SAE 2010 vehicle uses a 6-speed manual sequential transmission to transmit power from the engine to the drive train. As in other types of manual transmissions, the sequential transmission works by engaging and disengaging several sets of gears with shift forks to obtain different gear ratios. A cast steel drum inside the transmission, called the \emph{shift drum}, has a series of complex grooves machined into it, in which the shift forks ride. Rotating the drum causes the forks to engage one gear and disengage the other. A ratcheting mechanism allows a single external lever, called the \emph{shift lever}, to rotate the drum forward and back by a discrete amount. Each full movement of the lever forwards or backwards causes a gear change up or down, respectively \cite{HowtoManualTransmission, cbr600}.

%Brief {}``what is transmission''
%Shifting speed and dexterity required
%Stall avoidance
%-one chance to restart during race
%-sensitivity to stalling
%launching the car, etc.

\subsection{Clutch}

The clutch is actuated by a lever attached to the engine body. A pre-tensioned spring keeps the clutch engaged when there is no force on the lever. As force is applied to the lever, the clutch is gradually disengaged. The relationship between lever position the distance between the clutch plate and flywheel is non-linear. 

\subsection{Gear Selector}

The gear selector is actuated by a lever attached to the engine body. The gear selector may be rotated in either direction from its rest point. Up-shifting is accomplished by rotating the lever in one direction, while down-shifting is accomplished by rotating the lever in the opposite direction. 

\subsection{Gear Position Sensing}

\subsection{Neutral Sensing}

\subsection{Previous Implementations and Shortcomings}

There are two major downfalls to operating a purely manual transmission: it requires a great deal of dexterity and effort on behalf of the driver, and shifting time is relatively slow (around a second) relative to an electronically-assisted transmission (under one-hundred milliseconds.) For this reason, the Formula SAE vehicle switched to an electronically-assisted transmission system in the 2009 competition year. This system used electropneumatic actuators to engage the clutch and actuate the gear selector, and paddle-shifters to enable gear selection.

The electropneumatic system relieved the driver from the burden of manually engaging the clutch. It also reduced shift-time by a factor of nearly ten. However, the system made use of binary pneumatic valves which reduced the controllability of the clutch actuation. The rate at which the clutch was engaged was limited by the air flow rate of the valve itself. This resulted in a rough shifting sequence that was mechanically hard on the transmission. 
