\section{Engine and Transmission Module}


\subsection{Purpose}

Provide intelligent control of the intake runner length and transmission
system.

Provide the control signals required to actuate the clutch and gear
position levers with the pneumatic system.


\subsection{Processes}


\subsubsection{Variable Intake}


\subsubsection{Gear Selection}


\subsubsection{Advanced Transmission Features}


\subsubsection{Engine Starting/Stopping}

The starter solenoid is energized by signals from the engine controller. The driver may choose between an \emph{automatic} or \emph{manual} starting sequence. The \emph{driver interface module} (described in section \ref{sec:driver_interface_system}) relays the particular starting sequence command to the engine controller through the CAN bus.

\begin{figure}[H]
\centering
%\input{Figures/starter_system_overview}
\caption{Starter system overview.}
\label{fig:starter_system_overview}
\end{figure}

The \emph{automatic} starting sequence powers the starter solenoid until either the engine starts or a timeout elapses. In this case, the timeout is five seconds. The \emph{manual} starting sequence will engage the starter for as long as the driver demands, much like holding the key in the start position in a standard consumer automobile ignition system. 

The starter feature is disabled while the engine is running to avoid damaging the vehicle. The engine state is determined by monitoring the current RPM.

\subsection{Software}

Software overview diagram


\subsubsection{Transmission Manager}

Listens to transmission requests from the driver over the network.

Uses the event scheduler to sequence the complex series of actuation
vectors required by upshifting and downshifting.

Implement the PID feedback controller used to actuate the pneumatic
system.

Interacts with the PWM generator to create the control signals required
to actuate the pneumatic system.


\subsubsection{Intake Manager}

Continously monitor engine RPM and throttle position through messages
from the ECU.

Will adjust intake runner length based on a functional map of RPM
and throttle position.

Use hysterisis to avoid instability.

Map will be generated through dynometer testing.


\subsubsection{Starter Manager}

Listen for driver requests to start the engine.

Provide a means of one-touch starting -- sequences the actual starting
of the engine through the solenoid driver. 


\subsubsection{Event Scheduler}

Allow for complex sequencing of events in time.

Controllers may schedule new events to occur at various points in
time.

Scheduler will continuously update the schedule and signal the main
control loop to execute events that are now current.


\subsubsection{CAN Interface}

Provide a means of interfacing with the physical bus. 

Allow direction of particular messages to particular modules.


\subsubsection{PWM Generator}

Generate PWM signals of at least 20 Hz with a resolution of 1\% duty
cycle.

Two-channel design


\subsubsection{Main Control Loop}

Initializes the system and brings into a known state.

Waits for pending events to be executed and executes them.

Monitor for system faults and react accordingly.


\subsection{Hardware}

Show system hardware diagram


\subsubsection{Microcontroller}

Execute system control software.

In-circuit programmable and debuggable.

Has built-in CAN controller.

Has built-in RAM and ROM as well as EEPROM for holding configuration
parameters.


\subsubsection{CAN transceiver}

Interface module with the CAN bus.

Be capable of terminating the bus.


\subsubsection{High current solenoid drivers}

Controlling pneumatic solenoid valves (4)

Controlling starter solenoid


\subsubsection{Bidirectional I/O lines to the ECU}

Bidirectional interface to: launch control, shift cut, traction cut
percent, traction control on/off, traction control wet/dry.