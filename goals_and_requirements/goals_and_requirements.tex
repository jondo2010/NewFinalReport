\chapter{System Goals and Requirements\label{cha:goals_and_requirements}}

\section{Transmission and drivetrain}

The overall goal for the transmission and drivetrain was to improve over the work done in previous years. A first step was to refine and better understand the requirements of the current system, and then target improvements for the new one.

For successful transmission operations, \citet{clutch_control} quantify two requirements to be satisfied by any controller:
\begin{itemize}
  \item a \emph{no-kill} condition, whereby the engine speed must remain above a certain value $\omega_e^{min}$ and
  \item a \emph{no-lurch} condition which states that the derivative of the clutch slip $\dot{\omega}_{sl}=\dot{\omega}_e-\dot{\omega}_c$ at the moment of full engagement is near zero. This means the car doesn't jerk or lurch forward, and is a more important requirement when crawling than launching.
\end{itemize}

\nomenclature{$\omega_e^{min}$}{Minimum engine speed before a stall is likely.}
\nomenclature{$\omega_{sl}$}{The clutch slip, defined as the difference in speed between the engine and the clutch plates.}

\subsection{Shifting}

The electropneumatic subsystem should decrease shift timing to below \unit{100}{\milli\second} by incorporating a gear position sensor into the transmission, allowing the controller to sense exactly when a shift has completed. Shifting should also use less air, since the valves will be open for less time, and be mechanically less straining on the transmission.

The average force required to actuate the shift lever on the CBR600f4i was measured using a fish scale to be \unit{5.42}{\newton\metre}, including a factor of safety. Any actuation method designed will need to be able to produce this torque.

\subsection{Clutch control}

The electropneumatic subsystem should completely automate all clutching operations. For this, the pneumatic system will have to be tweaked, and the controller will need to incorporate several layers of feedback, including clutch position and various parameters from the ECU.

The transmission controller should be able to perform the following operations:
\begin{enumerate}
  \item Launch the car from a standstill by interacting with the ECU to enable the ECU's launch control feature, and engage the clutch in such a way that meets the \emph{no-kill} and to a lesser extent the \emph{no-lurch} conditions.
  \item Crawl the car from a standstill under \unit{25}{\kilo\metre\per\hour}, as well as be able to transition to regular driving, and back to a standstill while meeting the \emph{no-kill} and \emph{no-lurch} conditions.
  \item Implement a \emph{Neutral Find} feature that will automatically downshift from the current gear to neutral in the shortest amount of time.
  \item Implement an anti-stall feature that fulfills the \emph{no-kill} condition in the event of a spin-out by activating the Neutral Find.
  \item Implement an auto-upshift feature that will upshift the transmission without driver intervention to obtain the best possible acceleration.
\end{enumerate}

Additionally, all of these features should be tunable at run time. It should be possible to turn individual features on and off, and the controller should integrate with the networks VDM system.

The average force required to actuate the clutch lever on the CBR600f4i was measured using a torque wrench to be \unit{7.34}{\newton\metre}, including a factor of safety. Any actuation method designed will need to be able to produce this torque.

\section{Intake}

The goal of the intake subsystem is to implement an electronic controller for a variable-length (2-stage intake) intake plenum that will be actuated with a light-weight RC servomotor.

The controller must be able to react to changes in engine RPM, and select the best runner length for that RPM range.

Engine RPM ramp-rates under expected loading conditions were estimated with the engine on a dynamometer by the Formula SAE engine team. They have given us a timing requirement that the controller and actuator must be able to switch runner length configurations in approximately \unit{150}{\milli\second}.

The intake controller designed must be able to receive RPM data from the ECU over CANBus, and should be able to control an RC-Servomotor.

\section{Starter}

The goal of the starter subsystem is to ease control of the engine starting operation on the driver.

The starter subsystem will implement a dual-mode starting sequence, including manual and one-touch.

\section{Braking}


\subsection{Bias Adjustment}

Eliminate manual adjustment by implementing an electronic adjustment
capability. 

Actuate a 4-wire stepper motor attached to the brake pedal assembly.

Be able to actuate bias from the cockpit electronically from 30/70
to 70/30 by steps of 1/2\% in less than one second while vehicle in
motion.

Lock out bias adjustment while braking.


\subsection{Calibration}

Must provide ability to recalibrate adjustment systems to adjust for
mechanical wear and inconsistency. Calibration routines should take
no longer than 30 seconds. 


\section{Telemetry}

\subsection{Data Interfaces}

\subsubsection{ECU}

Provide a wireless pseudoserial link between ECU software and the
ECU module.

Two-way serial link running at 57.6 kbps.

Completely transparent to software.


\subsubsection{DAC}

Provide a wireless pseduoserial link between DAC software and the
DAC module.

One-way serial link running at 38.4 kbps.

Completely transparent to software.

Provide a means of injecting secondary data into the DAC stream, such
as engine RPM and other parameters, for monitoring by the DAC software.

Provide a means of decoding the DAC stream on the vehicle itself to
display information to the driver.


\subsubsection{Wireless Data Link}

Be able to interface with two laptops with a range of at least one
kilometre.

Provide on-the-fly resolution of interference conflicts with other
teams running similar wireless systems.

Provide wireless data health information by indicating wireless signal
strength as a percentage with resolution of 1\% of maximum signal
strength, indicating the number of packet errors occured since last
reset, etc.


\section{Driver Interface}


\subsection{Driver Controls}

Provide a means of shifting gears with minimum driver effort. (No
manual timing or clutching)

Provide a means of actuating all the features of the transmission
system (....)


\subsection{Vehicle Dynamic Adjustment}

Provide an easy to use interface to adjust vehicle dynamic parameters.

Allow driver to choose between sets of dynamic vehicle parameters
or `operating modes'.

Adjustment should be made by a single knob to reduce driver effort
required.

Adjustment of any vehicle parameters is verboten above 25 km/h.

Tuning of individual dynamic parameters and permanent changing of
preset modes should be capable through the driver interface itself.


\subsection{Diagnostic Information}

Critical warnings, such as engine overheating, low oil pressure, low battery, and other subsystem errors should be displayed to the user in a manner that will attract their attention.

Supplimentary vehicle performance information should be available to the driver (engine RPM, fuel, oil pressure, battery voltage, etc.) without being a distraction.


\subsection{Visually Appealing Interface}

Display startup logo.

Display easy-to-read fonts.

Interface should never be a distraction to the driver.

Visible in full light as well as low light.

Visible through sun visor helmet.


\section{Electrical System and Harness}

Minimize wiring, noise resiliance, maximise interconnectivity

power usage