\section{Engine and Transmission Module\label{sec:engine_transmission_design}}

The engine and transmission module provides optimized selection of the variable-length intake, changes gears at the driver's request without requiring manual clutching, and provides transmission features that make driving easier. Figure \ref{fig:design_engine_overview_block} shows an overview of the engine and transmission module and it's interactions with the environment.

\begin{figure}[H]
	\centering
 	\begin{tikzpicture}[auto, node distance=2cm, draw=black!70, >=stealth', font=\footnotesize]
  \node [block, blue shiny, minimum width=4cm, text width=4cm, right=-1cm, node distance=1.5cm] (module) {Braking Module};

  \node at ($(module.east)+(1cm,0)$) [block, text width=0.8cm] (ecu) {ECU};

  \node [bus, above of=module, node distance=1cm] (bus1) {};
  \node [bus, above of=ecu, node distance=1cm] (bus2) {};

  \draw [<->, thick] (ecu) to (module);

  \draw [-, line width=3pt] (bus1) --  ++(-1cm,0);
  \draw [-, line width=3pt] (bus1) -- node[label=above:CAN Bus]{} (bus2) -- ++(1cm, 0);
  \draw [<->, thick] (module) -- (bus1);
  \draw [<->, thick] (ecu) -- (bus2);

  \node at ($(module.west)!0.5!(module.south)+(0,-1.25cm)$) [red shiny, circle, label={[text width=1.5cm, rotate=90, below=5pt, left=10pt]below right:Servo}] (motor) {M};
  \node [block, left of=motor, node distance=1.5cm, text width=1cm] (intake) {Intake};

  \draw [<-, thick] (intake) -- (motor);
  \draw [<-, thick] (motor) -- ($(module.south west)!(motor.north)!(module.south)$);

  \node [block, below of=module, inner xsep=0pt, right=0cm, node distance=1.5cm] (pneumatics) {Electro-pneumatic system};
  \draw [<-, thick] (pneumatics) -- ($(module.south west)!(pneumatics.north)!(module.south)$);

  \node [block, right of=pneumatics, text width=2cm, node distance=3cm, above=0.25cm] (shift) {Shift lever};
  \node [block, right of=pneumatics, text width=2cm, node distance=3cm, below=0.25cm] (clutch) {Clutch lever};

  \draw [<-, thick, dashed] (clutch) -- ($(pneumatics.north east)!(clutch.west)!(pneumatics.south east)$);
  \draw [<-, thick, dashed] (shift) -- ($(pneumatics.north east)!(shift.west)!(pneumatics.south east)$);

  %%% Legend

  \draw [->, thick] ($(shift.east)+(2cm,0cm)$) -- ++(0.5cm,0) node [label={[font=\tiny]below:Electrical}] {} -- ++(0.5cm,0);
  \draw [->, thick, dashed] ($(clutch.east)+(2cm,0cm)$) -- ++(0.5cm,0) node [label={[font=\tiny]below:Mechanical}] {} -- ++ (0.5cm,0);
\end{tikzpicture}

	\caption{Overview of the engine and transmission module and it's environment.}
	\label{fig:design_engine_overview_block}
\end{figure}

The intake runner position is mechanically actuated with a \emph{servo motor}. The module generates the control signals required by electro-pneumatic system to actuate both the clutch and shift levers. The ECU interfaces with the module both directly through discrete inputs and through the CAN bus.

\subsection{Driver-Initiated Processes}

The major processes of the engine and transmission module are \emph{up-shifting}, \emph{down-shifting}, and \emph{neutral find}. They are described at a high level using flow charts. Requests to perform these processes are broadcast over the network from the driver interface. The module listens for these requests and executes the correct process accordingly.

\subsubsection{Up-Shifting}

The up-shift procedure is illustrated in \ref{fig:transmission_upshift_flow}. Up-shifting does not require disengaging the clutch, merely a small decrease in engine RPM. The ECU provides a shift-cut feature to cut spark to the engine during a shift operation to avoid the driver having to manually modulate the throttle during a shift.

\begin{figure}[H]
	\centering
	\begin{tikzpicture}[auto, node distance=2cm, draw=black!70, >=stealth', font=\scriptsize]
  \node[start, text width=1.2cm] (start) {Upshift Request};
  \node [decision, right of=start, text width=1cm, inner sep=0pt, node distance=2.5cm] (at_top) {Top\\gear?};
  \node [end, below of=at_top] (done) {Done};
  \node [block, right of=at_top, text width=1.5cm, node distance=2.5cm] (shiftcut) {Enable shiftcut};

  \node [block, right of=shiftcut, text width=1.5cm, node distance=2.5cm] (wait_shiftcut) {Wait for RPM drop below $RPM^{TH}_{cut}$};
  \node [block, right of=wait_shiftcut, text width=1.5cm, node distance=2.5cm] (upshift) {Engage upshift solenoid};
  \node [block, right of=upshift, text width=1.5cm, node distance=2.5cm] (wait_upshift) {Wait for shift feedback};
  \node [block, below of=wait_upshift, text width=1.5cm] (upshift_off) {Disengage upshift solenoid};
  \node [block, left of=upshift_off, text width=1.5cm, node distance=2.5cm] (shiftcut_off) {Disable shiftcut};

  \draw [->, thick] (start) -- (at_top);
  \draw [->, thick] (at_top) -- node[]{yes} (done);
  \draw [->, thick] (at_top) -- node[]{no} (shiftcut);

  \draw [->, thick] (shiftcut) -- (wait_shiftcut);
  \draw [->, thick] (wait_shiftcut) -- (upshift);
  \draw [->, thick] (upshift) -- (wait_upshift);
  \draw [->, thick] (wait_upshift) -- (upshift_off);
  \draw [->, thick] (upshift_off) -- (shiftcut_off);
  \draw [->, thick] (shiftcut_off) -- (done);
\end{tikzpicture}

	\caption{Flow-chart for the transmission up-shift procedure.}
	\label{fig:transmission_upshift_flow}
\end{figure}

The up-shift process depends upon engine RPM and gear position, both of which are provided by the ECU. The shift-cut feature is engaged on the ECU to drop the engine RPM by a small amount known as the cut RPM threshold, or $RPM^{TH}_{cut}$. The exact value of $RPM^{TH}_{cut}$ can be tuned on the ECU. Once the RPM reaches the cut threshold, the upshift solenoid is engaged, which actuates the pneumatics to push the shift lever. Once the module receives notification that the current gear has incremented, the pneumatics are relaxed, and shift cut feature is disengaged.

\nomenclature{$RPM^{TH}_{cut}$}{The threshold RPM is expected to drop when shift-cut is engaged for no-lift up-shifting.}

\subsubsection{Down-Shifting}

The down-shift procedure is illustrated in \ref{fig:transmission_downshift_flow}. The downshift procedure differs from the upshift procedure in that the downshift requires use of the clutch. 

\begin{figure}[H]
	\centering
	\begin{tikzpicture}[auto, node distance=2cm, draw=black!70, >=stealth', font=\scriptsize]
  \node[start, text width=1.4cm] (start) {Downshift Request};
  \node [decision, right of=start, text width=1cm, inner sep=0pt, node distance=2.5cm] (in_neutral) {Neutral?};
  \node [end, below of=in_neutral] (done) {Done};
  \node [block, right of=in_neutral, text width=1.5cm, node distance=2.5cm] (clutchin) {Disengage clutch};

  \node [block, right of=clutchin, text width=1.5cm, node distance=2.5cm] (downshift) {Engage downshift solenoid};
  \node [block, right of=downshift, text width=1.5cm, node distance=2.5cm] (wait_downshift) {Wait for shift feedback};
  \node [block, right of=wait_downshift, text width=1.5cm, node distance=2.5cm] (downshift_off) {Disengage downshift solenoid};
  \node [block, below of=downshift_off, text width=1.5cm] (wait_clutchin) {Wait for clutch-in request};
  \node [block, left of=wait_clutchin, text width=1.5cm, node distance=2.5cm] (clutchout) {Engage clutch};

  \draw [->, thick] (start) -- (in_neutral);
  \draw [->, thick] (in_neutral) -- node[]{yes} (done);
  \draw [->, thick] (in_neutral) -- node[]{no} (clutchin);

  \draw [->, thick] (clutchin) -- (downshift);
  \draw [->, thick] (downshift) -- (wait_downshift);
  \draw [->, thick] (wait_downshift) -- (downshift_off);
  \draw [->, thick] (downshift_off) -- (wait_clutchin);
  \draw [->, thick] (wait_clutchin) -- (clutchout);

  \draw [->, thick] (clutchout) -- (done);
\end{tikzpicture}

	\caption{Flow-chart for the transmission down-shift procedure.}
	\label{fig:transmission_downshift_flow}
\end{figure}

From the drivers perspective, down-shifting is a two-stage process:

\begin{enumerate}
  \item The driver pulls and holds the down-shift paddle on the steering wheel to request the gear change; and
  \item The driver releases the down-shift paddle to requests the clutch re-engage.
\end{enumerate}

Adding this extra clutch engagement request allows the driver to delay re-engagement of the clutch so that they can blip the throttle to avoid engine compression and the possibility of a spinout when the clutch plates re-engage\footnote{"Blipping" was previously described in Sec. \ref{sec:background_transmission}.}. 

Once the driver requests a down-shift, the clutch is disengaged and the down-shift solenoid is engaged to change gears. Once the correct gear has engaged, the solenoid is disengaged and the controller waits for the driver to release the down-shift paddle. When the paddle is released, the clutch is re-engaged. A minimum clutch disengagement time $t_{clutch_{min}}$ is observed in the event the driver pulls the paddle and releases it without any delay. This parameter is tunable.

\nomenclature{$t^{clutch}_{min}$}{Minimum clutch disengagement time during a downshift request.}

\subsubsection{Neutral Find}

Neutral find can be used by the driver to quickly shift into neutral when entering the pit area, without having to manually cycle through gears. The procedure is illustrated in \ref{fig:transmission_neutralfind_flow}.

\begin{figure}[H]
	\centering
	\begin{tikzpicture}[auto, node distance=2cm, draw=black!70, >=stealth', font=\scriptsize]
  \node[start, text width=1.4cm] (start) {Start};
  \node [decision, right of=start, text width=1.2cm, inner sep=0pt, node distance=2.5cm] (in_neutral1) {Neutral?};
  \node [end, below of=start, node distance=1.5cm] (done) {Done};
  \node [block, right of=in_neutral1, text width=1.5cm, node distance=2.5cm] (clutchin) {Disengage clutch};
  \node [block, right of=clutchin, text width=1.5cm, node distance=2.5cm] (downshift) {Downshift};
  \node [decision, right of=downshift, text width=1.2cm, inner sep=0pt, node distance=2.5cm] (in_neutral2) {Neutral?};

  \node [block, below of=clutchin, text width=1.5cm, node distance=1.5cm] (clutchout) {Engage clutch};

  \draw [->, thick] (in_neutral1.south) -- node[]{yes} (in_neutral1.south |- clutchout);

  \draw [->, thick] (start) -- (in_neutral1);
  \draw [->, thick] (in_neutral1) -- node[]{no} (clutchin);
  \draw [->, thick] (clutchin) -- node[coordinate](x1){} (downshift);
  \draw [->, thick] (downshift) -- (in_neutral2);
  \draw [->, thick] (in_neutral2.north) -- node[above]{no} ++(0,0.25cm) -| (x1);

  \draw [->, thick] (in_neutral2.south) |- node[]{yes} (clutchout);
  \draw [->, thick] (clutchout) -- (done);
\end{tikzpicture}

	\caption{Flow-chart for the neutral find procedure.}
	\label{fig:transmission_neutralfind_flow}
\end{figure}

When the driver engages the neutral find feature, the transmission controller disengages the clutch and downshifts until the neutral sensor indicates the transmission has reached the neutral gear. The clutch lever is then released with the transmission in neutral. If the driver attempts to engage the feature while the vehicle is already in neutral, the procedure is aborted.

\subsection{Variable Intake}

The length of the intake runners is modulated automatically to provide the optimal torque output for the current engine RPM. The control flow chart is shown in Fig. \ref{fig:engine_varintake_flow}. 
\begin{figure}[H]
	\centering
	\begin{tikzpicture}[auto, node distance=2cm, draw=black!70, >=stealth', font=\scriptsize]
  \node [start, text width=1.4cm] (start) {Start};
  \node [block, right of=start, text width=1.5cm, node distance=2.5cm] (update) {Update RPM};
  \node [decision, right of=update, text width=1cm, inner sep=0pt, node distance=2.5cm] (ideal) {Ideal length?};

  \node [block, right of=ideal, text width=1.5cm, node distance=2.5cm] (switch) {Switch lengths};
  \node [block, right of=switch, text width=1.5cm, node distance=2.5cm] (wait) {Adjust crossover};

  \draw [->, thick] (start) -- (update);
  \draw [->, thick] (update) -- (ideal);
  \draw [->, thick] (ideal) -- node[above]{no} (switch);
  \draw [->, thick] (switch) -- (wait);
  \draw [->, thick] (wait) -- ++(0, -1.5cm) node[coordinate](x1){} -| ($(start.east)!0.5!(update.west)$);
  \draw [->, thick] (ideal) -- node[]{yes} (ideal.south |- x1);

\end{tikzpicture}
	\caption{Flow-chart for the variable intake adjustment.}
	\label{fig:engine_varintake_flow}
\end{figure}

Optimizing the runner length requires coordination with the ECU to acquire the latest RPM values. A set-point RPM value to be determined will act as a cross-over point at which the runner length changes. As the engine RPM crosses over this line, the runner length is modulated. A brief timeout period exists to prevent the runner length from oscillating if the driver is operating near the cross-over point.

\subsection{Advanced Transmission Features}

The transmission portion of the module implements several features to aide the driver. The driver can enable or disable these features from the driver interface. The module will listen for feature requests over the network and act accordingly.

\subsubsection{Auto Up-shift}

This feature of the engine module is aimed primarily at improving performance in the acceleration event. Based on known torque curves, a table of optimal shift points in the RPM range is developed. As the engine reaches the top RPM for a given gear, the engine module will automatically upshift to the next gear, without any driver input. All the driver needs to do is maintain full throttle, and hang on.

\subsubsection{Launch}

Full-throttle launch uses an important feature of the ECU called launch control. From a stand-still, the slip ratio of the driven wheels to the non-driven wheels is monitored, and the engine output power is reduced until the ratio reaches 1:1. Drivers use this feature by maintaining full-throttle at the starting line while holding the brake pedal. As soon as the brake pedal is released, the the engine module will release the clutch in a controlled manner in an attempt to get the best possible acceleration.

\subsubsection{Crawl}

Part-throttle launch is a feature designed to mimic an automatic transmission. By controlling the clutch position, and thereby modulating the amount of torque transferred to the wheels for a short period of time, the car can be made to creep slowly from a stand-still. This will be used when driving up to the starting line of various dynamic events.

\subsection{Hardware \label{sec:design_engine_transmission_hardware}}

A high-level overview of the module's hardware design is shown in figure \ref{fig:engine_hardware_design_block}. The module makes use of a \emph{micro-controller} to execute all of the controller software. 

\vspace{1em}
\begin{figure}[H]
	\centering
	\begin{tikzpicture}[auto, node distance=2cm, draw=black!70, >=stealth']

  \node [block, font=\scriptsize, text width=1.5cm, inner xsep=0, minimum height=1cm] (intake) {Intake};
  \node [red shiny, circle, below of=intake, node distance=1.15cm, inner sep=1pt, label=right:Servo] (servo) {M};

  \draw [->, dashed, thick] (servo) -- (intake);

  %%% Balance bar and EOT switches
  \node [block, right of=intake, font=\scriptsize, text width=1.5cm, node distance=3cm, inner xsep=0, minimum height=1cm] (clutch) {Clutch Actuator};
  \node [block, right of=clutch, font=\scriptsize, text width=1.35cm, inner xsep=0, minimum height=1cm] (clutch_solenoids) {Clutch Solenoids};
  \node [block, right of=clutch_solenoids, font=\scriptsize, text width=1.35cm, inner xsep=0, minimum height=1cm] (shift_solenoids) {Shift Solenoids};
  \node [block, right of=shift_solenoids, font=\scriptsize, text width=1.5cm, inner xsep=0, minimum height=1cm] (shifter) {Shift Actuator};

  \node at ($(clutch_solenoids)!0.5!(shift_solenoids)+(0,-2cm)$)
    [block, node distance=1cm, text width=2.5cm, font=\scriptsize, minimum width=3cm] (driver) {Solenoid Drivers};

  \node [block, right of=shifter, font=\scriptsize, text width=1.5cm, inner xsep=0, minimum height=1cm, node distance=3cm] (ecu) {ECU};

  \draw [<-, thick] (clutch_solenoids) -- ($(driver.north west)!(clutch_solenoids.south)!(driver.north east)$);
  \draw [<-, thick] (shift_solenoids) -- ($(driver.north west)!(shift_solenoids.south)!(driver.north east)$);
  \draw [->, dashed, thick] (clutch_solenoids) to (clutch);
  \draw [->, dashed, thick] (shift_solenoids) to (shifter);

  \draw [->, thick] (shifter) to (ecu);

  \node [block, below of=clutch, text width=1.5cm] (adc) {ADC};
  \draw [->, thick] (clutch) to (adc);

  %%% Microcontroller block
  \path ($(intake.south west |- adc.south west) + (0,-1cm)$) node (mu1) {};
  \path ($(shifter.south east |- driver.south east) + (0,-1cm)$) node (mu2) {};

  \node [block, fit=(mu1) (mu2), inner xsep=0, minimum height=1cm] (micro) {Microcontroller};

   \draw [<-, thick] (servo.south) -- ($(micro.north west)!(servo.south)!(micro.north east)$);
   \draw [->, thick] (adc.south) -- ($(micro.north west)!(adc.south)!(micro.north east)$);
   \draw [<-, thick] (driver.south) -- ($(micro.north west)!(driver.south)!(micro.north east)$);

% 
%   \draw [->, thick] (eot1.south) -- ($(micro.north west)!(eot1.south)!(micro.north east)$);
%   \draw [->, thick] (eot2.south) -- ($(micro.north west)!(eot2.south)!(micro.north east)$);
% 
%   \draw [<-, thick] (driver.south) -- ($(micro.north west)!(driver.south)!(micro.north east)$);

  \draw [<-, thick] (ecu.south) |- ($(shifter.south) + (0,-0.5cm)$) node[name=t]{} -- ($(micro.north west)!(t)!(micro.north east)$);

  %%% CAN Bus

  \node at ($(micro.east)+(1.5cm,0)$) [block, name=can, inner xsep=2pt] {CAN Transceiver};
  \draw [<->, thick] (micro) to (can);
  \draw [-, line width=3pt] (ecu.east) -- ++(0.5cm,0) |- node[name=can1,coordinate,label={[rotate=90]below:CAN Bus}]{} (can);
  \draw [-, line width=3pt] (can1) -- ++(0,-1cm);

  %%% Background

   \begin{pgfonlayer}{background}
     \path (intake.south west |- adc.north west)+(-0.3,0.3) node (a) {};
     \path (can.south east)+(+0.2,-0.2) node (b) {};
     \path[module] (a) rectangle (b);
   \end{pgfonlayer}

  %%% Legend

  \draw [->, thick] ($(micro.south west)+(0.5cm,-0.5cm)$) -- ++(0.5cm,0) node [label={[font=\tiny]below:Electrical}] {} -- ++(0.5cm,0);
  \draw [->, thick, dashed] ($(micro.south west)+(2cm,-0.5cm)$) -- ++(0.5cm,0) node [label={[font=\tiny]below:Mechanical}] {} -- ++ (0.5cm,0);
\end{tikzpicture}

	\caption{Block diagram of the engine and transmission module hardware.}
	\label{fig:engine_hardware_design_block}
\end{figure}

\nomenclature{ADC}{Analog-to-Digital Converter}
\nomenclature{GPIO}{General Purpose Input-Output}

Inter-module communication over the CAN bus is mediated with a \emph{CAN transceiver}. An \emph{analog-to-digital converter} (ADC) samples the position of the clutch. Unique to this module are high-current \emph{solenoid drivers} that are used to engage the solenoid valves that actuate the clutch and shift levers. Discrete signals to the ECU are connected to \emph{general purpose input-output} (GPIO) lines of the micro-controller.

\subsubsection{Micro-controller}

\nomenclature{EEPROM}{Electronically-Erasable Programmable Read-Only Memory}

The main purpose of the micro-controller is to execute system control software. The micro-controller should be in-circuit programmable and debuggable to speed development, with a built-in CAN controller. It should also have built-in \emph{electronically-erasable programmable read-only memory} (EEPROM) for holding configuration parameters.

\subsubsection{CAN Transceiver}

The CAN transceiver translates the digital bits generated by the CAN controller into a differential line-coded signal suitable for broadcasting over the bus. Both ends of a CAN bus must be terminated with a \unit{120}{\ohm} resistor \cite{MCP2551}. The module should be configurable to terminate the bus if desired.

\subsubsection{Analogue-to-Digital Converters}

The analogue-to-digital converters sample the clutch position sensor. The output voltage of the particular sensor used is \unit{0-5}{\volt}, corresponding to 0-100\% engagement of the clutch. The required sampling frequency is \unit{1}{\kilo\hertz}.

\subsubsection{Solenoid Drivers}

High-current solenoid drivers provide current to the solenoids that actuate the clutch and shift levers. The actual amount of current required depends on the particular implementation. We must use drivers that can source enough current to provide the torque required to actuate the clutch and shift levers. 
 
\subsubsection{ECU Control Signals}

The engine and transmission module design has several outputs to control the discrete input pins on the ECU as described in Sec. \ref{sec:ecu_background_discrete_inputs}:

\begin{itemize}
  \item the launch control pin, which enables and disables the launch control feature;
  \item the shift cut pin, which, when held high, enables shift cut;
  \item the traction cut percent pin, which is a \unit{0-5}{\volt} analog input that sets the amount of wheel slip threshold for traction control;
  \item the traction control on/off pin, which toggles the traction control feature; and
  \item the traction control wet/dry pin, which swaps between two traction control presets.
\end{itemize}

\subsection{Software}

A block-diagram overview of the software design is shown in Fig. \ref{fig:engine_software_design_block}. The functionality of the system is separated into an \emph{intake manager} and a \emph{transmission manager}. The managers interact with a set of \emph{hardware abstraction interfaces} to perform their tasks. A \emph{pulse width modulation (PWM) generator} provides the signals required to interact with the electro-pneumatic system. The entire process is overseen by a \emph{module coordinator}.

\begin{figure}[H]
	\centering
	\tikzstyle{big arrow} = [>=latex, line width=4pt, gray]

\begin{tikzpicture}[auto, node distance=2cm, draw=black!70, >=stealth', font=\scriptsize]
  %\draw[help lines] (-1,-5) grid (8,2);
  
  \node [block, minimum width=2cm, inner xsep=0] (servo) {Servo Interface};

  \node [block, grey shiny, minimum width=5cm, inner xsep=0, right of=servo, right=-1cm, text width=5cm, node distance=3cm] (intake_manager) {Intake Manager};

  \node [block, minimum width=2cm, inner xsep=0, below of=intake_manager, node distance=2cm, right=-2.5cm, minimum height=1cm] (can) {CAN Interface};
  \node [block, minimum width=2cm, inner xsep=0, below of=intake_manager, node distance=2cm, left=-2.5cm, minimum height=1cm] (nv) {NV Storage Interface};

  \node [block, grey shiny, minimum width=5cm, inner xsep=0, below of=intake_manager, text width=5cm, node distance=4cm] (transmission_manager) {Transmission Manager};

  \node [block, minimum width=2cm, inner xsep=0, below of=servo, node distance=4cm] (pwm) {PWM Generator};

  \node [block, minimum width=2cm, inner xsep=0, below of=transmission_manager, node distance=1.5cm, right=-2.5cm, minimum height=1cm] (adc) {ADC Interface};
  \node [block, minimum width=2cm, inner xsep=0, below of=transmission_manager, node distance=1.5cm, left=-2.5cm, text width=1.5cm, minimum height=1cm] (gpio) {GPIO Interface};

  \node [block, minimum width=2cm, inner xsep=0, right of=nv, node distance=3cm] (coordinator) {Module Coordinator};
  \node [block, minimum width=2cm, inner xsep=0, right of=coordinator, node distance=3cm] (scheduler) {Event Scheduler};

  \draw [<-, big arrow] (servo) -- (intake_manager);
  \draw [<-, big arrow] (pwm) -- (transmission_manager);

  \draw [<->, big arrow] (can.north) -- ($(intake_manager.south west)!(can.north)!(intake_manager.south east)$);
  \draw [<->, big arrow] (nv.north) -- ($(intake_manager.south west)!(nv.north)!(intake_manager.south east)$);

  \draw [<->, big arrow] (can.south) -- ($(transmission_manager.north west)!(can.south)!(transmission_manager.north east)$);
  \draw [<->, big arrow] (nv.south) -- ($(transmission_manager.north west)!(nv.south)!(transmission_manager.north east)$);

  \draw [<-, big arrow] (adc.north) -- ($(transmission_manager.south west)!(adc.north)!(transmission_manager.south east)$);
  \draw [->, big arrow] (gpio.north) -- ($(transmission_manager.south west)!(gpio.north)!(transmission_manager.south east)$);

  \draw [<->, big arrow] (transmission_manager) -| node[](x1){} (coordinator);
  \draw [->, big arrow] (x1) -| (scheduler);
  \draw [<->, big arrow] (coordinator) -- (scheduler);

  \draw [<->, big arrow] (intake_manager) -| (coordinator);
 
\end{tikzpicture}

	\caption{Block diagram of the engine and transmission module software.}
	\label{fig:engine_software_design_block}
\end{figure}

The external hardware and micro-controller features that are needed by the intake and pressure managers are best abstracted through \emph{hardware abstraction interfaces}. This allows for a modular design that is not intimately mated with any particular external hardware choice. 

\subsubsection{Transmission Manager}

The transmission manager listens to shifting and feature requests from the driver over the network. It implements the up-shift, down-shift, and neutral find processes. It is also responsible for implementing the advanced transmission features such as launch and crawl.

The transmission manager uses the event scheduler to sequence the complex series of actuation vectors required by up-shifting and down-shifting. It also implements the feedback control system required by the electro-pneumatic system to smoothly actuate the clutch lever. Timing signals are generated with the help of the PWM generator.

\subsubsection{Intake Manager}

The intake manager continuously monitors the engine RPM being broadcast by the ECU. It adjusts the intake runner length based on a cross-over point determined by dynometer testing. Adjustments to the runner length are made via the servo interface.

\subsubsection{Event Scheduler}

The event scheduler allows for complex sequencing of events in time. It allows for new events to be scheduled at some point in the future.
The scheduler will continuously update the schedule and signal the module coordinator to execute events that are now current. The scheduler requires a resolution of \emph{1}{\milli\second} to be useful to the transmission manager for scheduling solenoid actuation events.

\subsubsection{Pulse Width Modulation (PWM) Generator}

The PWM generator generates the signals required by the transmission manager to engage the clutch lever solenoids. To meet the goals of the transmission system, the signals must have a frequency of at least \unit{20}{\hertz} and a 1\% duty cycle resolution. Since there are two solenoids to be activated, the generator must be able to offer two independent output channels.

\subsubsection{Module Coordinator}

Each of the processes implemented by the managers are made up of several states. Asynchronous events, such as a request to change gears, are handled by the managers themselves. The module coordinator contains the low-level state machine mechanisms needed by the two managers to transition between their internal states. It keeps track of the current state of the system, and also initializes all of the hardware abstraction interfaces and managers at start-up time. The coordinator also interacts with the event scheduler, waiting for pending events to be executed and then executing them.

\subsubsection{CAN Interface}

The CAN interface enables network communication over the CAN bus. Managers can schedule packets for transmission, and incoming packets can be routed to the appropriate manager. It is common to all four modules.

\subsubsection{Solenoid Driver Interface}

The solenoid driver interface communicated with the high-current solenoid driver to switch the solenoid valves on and off. Logical requests from the transmission manager are translated into a format understandable by the driver.

\subsubsection{Servo Motor Interface}

The \emph{servo motor interface} takes a positional adjustment command from the program. It utilizes the pulse width modulation generator to generate a drive signal for the servo motor with the correct duty cycle and frequency.

\subsubsection{General-Purpose Input-Output (GPIO) Interface}
\label{sec:design_engine_gpio}

The GPIO interface provides a means of setting any of the control lines connected to the ECU. It is responsible for generating the correct logic levels for a given ECU feature request.

\subsubsection{Analog-to-Digital (ADC) Interface} 
\label{sec:design_engine_adc}

The ADC interface communicates with the hardware analog-to-digital converter. It can initiate a sampling cycle on either return the results to the program for further processing.

\nomenclature{NVRAM}{Non-Volatile Random-Access Memory}
\subsubsection{Non-Volatile Storage Interface}
\label{sec:design_engine_nvsi}

The non-volatile storage interface can read and write data to and from \emph{non-volatile random-access memory} (NVRAM). This enables configuration parameters to be stored and read even if the module loses power or is reset.
