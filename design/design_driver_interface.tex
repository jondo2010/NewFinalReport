\section{Driver Interface Module\label{sec:Driver-Interface-Module}}

\nomenclature{LCD}{Liquid Crystal Display}

In previous years, the team has often been hindered by a lack of direct information from the electronic systems in the car. The driver interface module remedies this by providing information to the driver in real-time, and allows the driver to control aspects of the electronic control systems.

The driver interface system consists of:

\begin{itemize}
\item \emph{driver controls}, which are tactile inputs for adjusting vehicle dynamics; 
\item the \emph{information display panel}, which displays vehicle information to the driver; and
\item the \emph{driver interface module} which collects information from the driver controls and other modules in the system and relays them to the driver through the information display panel, and also relays driver inputs to the other modules.
\end{itemize}


\subsection{Purpose}


\subsection{Processes}


\subsubsection{Driver Controls}
The driver controls consist of three rotary encoder knobs and three buttons. See table \ref{table:driver_controls} for a description of each.

\begin{table}[H]
\caption{Available driver controls.\label{table:driver_controls}}
\centering
\begin{tabular}{|c|c|p{8 cm}|}
	\hline 
	Type & Name & Description \\
	\hline
	\hline 
	Rotary & VDM & Changes the current VDM setting. \\
	\hline 
	Rotary & Option & Changes the currently selected menu option. \\
	\hline
	Rotary & Adjust & Adjusts the value associated with the currently selected menu option. \\
	\hline
	Button & Starter & Engages the automatic start sequence if pressed once, or the manual start sequence if held down.\\
	\hline
	Button & Neutral Find & Activates the neutral-find feature. \\
	\hline
	Button & Diagnostics/Save & Pages through diagnostic information if pressed once, or saves the current dynamic settings to the selected VDM if held down.\\		
	\hline 		
\end{tabular}
\end{table}


\subsubsection{Diagnostic Information}

The interface display panel is a large monochrome \emph{Liquid Crystal Display} (LCD) screen, inset into the steering wheel. The panel provides real-time information to the driver regarding all of the electronic systems in the car. Primary information displayed on the panel includes:

\begin{itemize}
\item the selected gear;
\item the active \emph{vehicle dynamics mode} (VDM);
\item the telemetry signal strength;
\item the engine RPM;
\item the vehicle wheel speed; and
\item the status of the launch control feature.
\end{itemize}

The list of available options and the current value for the selected option is displayed on screen when the driver rotates the option dial. After a timeout period, normal real-time information display resumes. 



\subsubsection{Vehicle Dynamics Mode (VDM)}
\nomenclature{VDM}{Vehicle Dynamics Mode}

The driver interface system provides a method of quickly modifying several dynamic vehicle parameters quickly and easily. For example, an acceleration event calls for launch control, auto-upshift, and a heavy forward brake bias. It is possible to enable all of these features in one step by changing the VDM mode to {}``acceleration''. 

When a new mode is selected, all nodes on the network are notified and synchronized to modify their dynamic parameters in accordance with the specific mode. For example, when the  {}``acceleration'' mode is enabled, the engine module will enable launch control and auto-upshift, and the brake controller will modify the brake bias to a pre-set ratio.

\begin{description}
  \item [{Pit Mode}] enables soft-launch driving characteristics that mimic a fully automatic transmission. This makes slowly driving the car forward from a stand-still far easier, and only requires the driver to take their left foot off the brake, and slightly apply the throttle.
  \item [{Acceleration Mode}] puts the vehicle systems into full-performance characteristics. Launch control is activated. The engine module will watch for a launch signal from the driver, and will automatically up-shift based on the engine RPM.
  \item [{Dynamic Mode}] puts the vehicle systems into a mode that is suitable for the autocross, and the endurance race.
\end{description}


\subsubsection{Parameter Tuning}


\subsection{Software}


\subsubsection{User Interface Manager}


\subsubsection{Driver Control Manager}


\subsubsection{Vehicle Diagnostics Manager}


\subsubsection{Vehicle Dynamic Mode Manager}


\subsubsection{I/O Monitor}


\subsubsection{CAN Interface}


\subsubsection{Main Control Loop}


\subsection{Hardware}


\subsubsection{Microcontroller}


\subsubsection{CAN Transceiver}


\subsubsection{LCD Module}


\subsubsection{Input Knobs and Buttons}


\subsubsection{Paddle Shifters}