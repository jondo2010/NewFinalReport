%% LyX 1.6.4 created this file.  For more info, see http://www.lyx.org/.
%% Do not edit unless you really know what you are doing.
\documentclass[english]{scrreprt}
\usepackage[T1]{fontenc}
\usepackage[latin9]{inputenc}
\setcounter{secnumdepth}{3}
\usepackage{babel}

\begin{document}
\tableofcontents{}


\chapter{Introduction}


\section{Formula SAE}


\section{Motivation}


\section{Problem Definition}


\section{Strategy}


\section{Outline of Thesis}


\chapter{Background}


\section{Transmission}


\subsection{Overview}

Brief {}``what is transmission''

Shifting speed and dexterity required

Stall avoidance

-one chance to restart during race

-sensitivity to stalling

launching the car, etc.


\subsection{Mechanical Systems}

Simple descriptions


\subsubsection{Clutch}


\subsubsection{Gear Selection}


\subsection{Previous Implementations and Shortcomings}

How, challenges, etc.


\subsubsection{Electropneumatic Actuation}


\subsubsection{Gear Position Sensing}


\subsubsection{Neutral Sensing}


\section{Engine}


\subsection{Overview}

brief {}``what is engine'', honda cbr, etc.

maximise power output (performance application)

torque power curve, etc.


\subsection{Intake and Exhaust}

Intake background, pressure waves, etc. Torque curve depends on length, 


\subsection{Research and Modelling of Variable Length Intake}

present research done by others on team, quantified runner length
dependence

intake length changes power

quantified length versus power on actual vehicle

chose optimal intake length

proposed variable length intake system for future work


\subsection{Starting System}

current requirements, starter motor, etc.


\section{Braking}


\subsection{Overview}

Idea of braking and force distribution


\subsection{Mechanical Systems}


\subsubsection{Independant Hydraulic Systems}


\subsubsection{Balance Bar}


\subsection{Adjustment Difficulties}

Competition problems, manual adjustment difficulty, scope of adjustment,
tuneability lag


\section{Telemetry}


\subsection{Overview}

Why is telemetry necessary? Sensors, data, etc.


\subsection{ECU}


\subsection{DAC}


\subsection{Previous Implementations and Shortcomings}


\subsubsection{Cellular Link}


\subsubsection{Off-the-Shelf XBee Link}

Both of these only worked with one device, weren't reliable, etc.


\section{Driver Interface}


\subsection{Overview}

Driver and crew needs easy access to data, easy control of on-board
systems


\subsection{Driver Controls}


\subsubsection{Transmission Control}

Upshift/Downshift

Activate other trans. control features, neutral find, etc., launch


\subsubsection{Engine Control}


\subsection{Vehicle Diagnostics}


\subsubsection{Critical Indicators}

Overheating, oil pressure, etc.


\subsubsection{Supplimentary Indicators}

information about telemetry, shift control, etc.


\subsection{Previous Implementations and Shortcomings}


\chapter{Goals and Requirements}


\section{Transmission}


\subsection{Shift Speed}

Improve shift speed and smoothness, 50 ms to upshift, 100 ms to downshift.


\subsection{Launch Control}

Provide launch control, decrease launch time and wheel slippage by
xxx.


\subsection{Neutral Find}

Decrease time required to find neutral, worst case is 500 ms.


\subsection{Anti-Stall}

Implement anti-stall feature, avoid stalling in any gear in spin-out
situations etc.


\subsection{Electropneumatics}

Reduce air consumption by 50\%, improve reliability. 


\section{Intake}

Implement variable-length (2-stage intake) within 150 ms to gain a
power advantage of xxx ftlbs torque over previous.

Automatically optimize runner length by monitoring throttle position,
engine RPM data from ECU.


\section{Starter}

Implement dual-mode starting system, including manual and one-touch.

Provide at least 5A of current for 30 seconds to starter solenoid.


\section{Braking}


\subsection{Bias Adjustment}

Eliminate manual adjustment by implementing an electronic adjustment
capability. 

Actuate a 4-wire stepper motor attached to the brake pedal assembly.

Be able to actuate bias from the cockpit electronically from 30/70
to 70/30 by steps of 1/2\% in less than one second while vehicle in
motion.

Lock out bias adjustment while braking.


\subsection{Calibration}

Must provide ability to recalibrate adjustment systems to adjust for
mechanical wear and inconsistency. Calibration routines should take
no longer than 30 seconds. 


\section{Telemetry}


\subsection{Data Interfaces}


\subsubsection{ECU }

Provide a wireless pseudoserial link between ECU software and the
ECU module.

Two-way serial link running at 57.6 kbps.

Completely transparent to software.


\subsubsection{DAC }

Provide a wireless pseduoserial link between DAC software and the
DAC module.

One-way serial link running at 38.4 kbps.

Completely transparent to software.

Provide a means of injecting secondary data into the DAC stream, such
as engine RPM and other parameters, for monitoring by the DAC software.

Provide a means of decoding the DAC stream on the vehicle itself to
display information to the driver.


\subsubsection{Wireless Data Link}

Be able to interface with two laptops with a range of at least one
kilometre.

Provide on-the-fly resolution of interference conflicts with other
teams running similar wireless systems.

Provide wireless data health information by indicating wireless signal
strength as a percentage with resolution of 1\% of maximum signal
strength, indicating the number of packet errors occured since last
reset, etc.


\section{Driver Interface}


\subsection{Driver Controls}

Provide a means of shifting gears with minimum driver effort. (No
manual timing or clutching)

Provide a means of actuating all the features of the transmission
system (....)


\subsection{Vehicle Dynamic Adjustment}

Provide an easy to use interface to adjust vehicle dynamic parameters.

Allow driver to choose between sets of dynamic vehicle parameters
or `operating modes'.

Adjustment should be made by a single knob to reduce driver effort
required.

Adjustment of any vehicle parameters is verboten above 25 km/h.

Tuning of individual dynamic parameters and permanent changing of
preset modes should be capable through the driver interface itself.


\subsection{Diagnostic Information}

Critical warnings (....) must be delivered to the driver on an easy-to-read
display.

Supplimentary vehicle performance information should be available
to the driver (engine RPM, fuel, oil pressure, battery voltage, etc.)
without being a distraction.


\subsection{Visually Appealing Interface}

Display startup logo.

Display easy-to-read fonts.

Interface should never be a distraction to the driver.

Visible in full light as well as low light.

Visible through sun visor helmet.


\section{Electrical System and Harness}

Minimize wiring, noise resiliance, maximise interconnectivity

power usage


\chapter{Design}


\section{Architectural Overview}

Picture of the system overview.

Electropneumatic transmission actuators

Engine and transmission module

Braking module

Telemetry module

Driver control module

Located at points closest to the mechanical systems they interact
with.

Design systems as modularly as possible to reduce development time.

CAN Bus to reduce the number of point-to-point wiring required and
increase noise resiliancy, and interface with the existing ECU system
easily.


\section{Electropneumatic Transmission Actuation}


\subsection{Purpose}

Actuation of the clutch and gear selection levers. 

Be controlled electronically by the transmission control module.


\subsection{System Overview}

Show pneumatic diagram.


\subsubsection{Air tank }


\subsubsection{Regulator }


\subsubsection{Valves }


\subsubsection{Cylinders }


\subsubsection{Mechanical Linkage }


\subsubsection{Positional Sensors }


\subsection{Valving Design}

Reduces air consumption

Increases controllability


\subsection{Modelling}

Entire system modelled in Simulink


\section{Engine and Transmission Module}


\subsection{Purpose}

Provide intelligent control of the intake runner length and transmission
system.

Provide the control signals required to actuate the clutch and gear
position levers with the pneumatic system.


\subsection{Processes}


\subsubsection{Variable Intake}


\subsubsection{Gear Selection}


\subsubsection{Advanced Transmission Features}


\subsubsection{Engine Starting/Stopping}


\subsection{Software}

Software overview diagram


\subsubsection{Transmission Manager}

Listens to transmission requests from the driver over the network.

Uses the event scheduler to sequence the complex series of actuation
vectors required by upshifting and downshifting.

Implement the PID feedback controller used to actuate the pneumatic
system.

Interacts with the PWM generator to create the control signals required
to actuate the pneumatic system.


\subsubsection{Intake Manager}

Continously monitor engine RPM and throttle position through messages
from the ECU.

Will adjust intake runner length based on a functional map of RPM
and throttle position.

Use hysterisis to avoid instability.

Map will be generated through dynometer testing.


\subsubsection{Starter Manager}

Listen for driver requests to start the engine.

Provide a means of one-touch starting -- sequences the actual starting
of the engine through the solenoid driver. 


\subsubsection{Event Scheduler}

Allow for complex sequencing of events in time.

Controllers may schedule new events to occur at various points in
time.

Scheduler will continuously update the schedule and signal the main
control loop to execute events that are now current.


\subsubsection{CAN Interface}

Provide a means of interfacing with the physical bus. 

Allow direction of particular messages to particular modules.


\subsubsection{PWM Generator}

Generate PWM signals of at least 20 Hz with a resolution of 1\% duty
cycle.

Two-channel design


\subsubsection{Main Control Loop}

Initializes the system and brings into a known state.

Waits for pending events to be executed and executes them.

Monitor for system faults and react accordingly.


\subsection{Hardware}

Show system hardware diagram


\subsubsection{Microcontroller}

Execute system control software.

In-circuit programmable and debuggable.

Has built-in CAN controller.

Has built-in RAM and ROM as well as EEPROM for holding configuration
parameters.


\subsubsection{CAN transceiver}

Interface module with the CAN bus.

Be capable of terminating the bus.


\subsubsection{High current solenoid drivers}

Controlling pneumatic solenoid valves (4)

Controlling starter solenoid


\subsubsection{Bidirectional I/O lines to the ECU}

Bidirectional interface to: launch control, shift cut, traction cut
percent, traction control on/off, traction control wet/dry.


\section{Braking Module Design\label{sec:Braking-Module-Design}}


\subsection{Purpose}

Adjust the brake bias position electronically and on-demand from the
driver.

Provide calibration sequences to allow for mechanical wear.


\subsection{Processes}


\subsubsection{Brake Bias Calibration}


\subsubsection{Brake Bias Adjustment}


\subsubsection{Pressure Sensor Calibration}


\subsection{Software}

Software overview diagram


\subsubsection{Bias Manager}

Handles requests to adjust the brake bias from the network. 

Performs balance bar calibration sequence on demand.


\subsubsection{Pressure Manager}

Continuously samples brake pressure.

Performs a pressure calibration sequence on demand.


\subsubsection{CAN Interface}

As documented above.


\subsubsection{Main Control Loop}

Initializes the system and brings into a known state.

Monitor for system faults and react accordingly.


\subsection{Hardware}


\subsubsection{Microcontroller}

As mentioned above.


\subsubsection{CAN transceiver}

As mentioned above.


\subsubsection{Stepper Motor Driver}

Generates signals required to actuate the stepper motor used to adjust
the balance bar.


\subsubsection{Analogue-to-Digital Converter}

Samples both front and rear pressure sensors. Capable of 10-bit resolution.


\subsubsection{End-of-Travel Sensing Switches}


\section{Telemetry Module Design\label{sec:Telemetry-Module-Design}}


\subsection{Purpose}

Provide a multiplexed virtual serial link between the ECU and the
ECU software, and the DAC and the DAC software. Additionally be able
to decode the DAC serial data on the car, and inject additional data
channels into the DAC serial stream.


\subsection{Processes}


\subsubsection{ECU Transmission Channel}


\subsubsection{DAC Transmission Channel}


\subsubsection{DAC Packet Capture and Injection}


\subsection{Software}


\subsubsection{Wireless Interface Manager }


\subsubsection{Data Multiplexor}


\subsubsection{Data Decoder }


\subsubsection{Packet Injector}


\subsubsection{CAN Interface}


\subsubsection{Main Control Loop}


\subsection{Hardware}


\subsubsection{Microcontroller}


\subsubsection{CAN Transceiver}


\subsubsection{High-Speed UART Bank}


\subsubsection{Wireless Modem}


\section{Driver Interface Module Design\label{sec:Driver-Interface-Module}}


\subsection{Purpose}


\subsection{Processes}


\subsubsection{Driver Controls}


\subsubsection{Diagnostic Information}


\subsubsection{Vehicle Dynamic Mode}


\subsubsection{Parameter Tuning}


\subsection{Software}


\subsubsection{User Interface Manager}


\subsubsection{Driver Control Manager}


\subsubsection{Vehicle Diagnostics Manager}


\subsubsection{Vehicle Dynamic Mode Manager}


\subsubsection{I/O Monitor}


\subsubsection{CAN Interface}


\subsubsection{Main Control Loop}


\subsection{Hardware}


\subsubsection{Microcontroller}


\subsubsection{CAN Transceiver}


\subsubsection{LCD Module}


\subsubsection{Input Knobs and Buttons}


\subsubsection{Paddle Shifters}


\chapter{Implementation\label{cha:Implementation}}


\section{Electropneumatic Implementation}


\subsection{Methodology}


\subsection{Solenoid Valves}


\subsection{Pneumatic Actuators}


\subsection{Positional Feedback Sensors}


\subsection{Mechanical Linkage}


\section{Hardware Implementation\label{sec:Hardware-Implementation}}


\subsection{Methodology}

Commonality between modules, same life-support system.


\subsection{CAD Design}

Schematic capture and PCB layout using EagleCAD.


\subsection{Commonalities }


\subsubsection{Microcontroller}


\subsubsection{CAN Transceiver}


\subsubsection{Linear Regulator}


\subsubsection{Supervisor}


\subsubsection{Wiring Harness}


\subsection{Engine and Transmission Module}

<Picture of board>

Overview of physical implementation.


\subsubsection{High Current Solenoid Driver}


\subsubsection{Input Buffers}


\subsection{Braking Module}

<Picture of board>


\subsubsection{Stepper Motor Driver}


\subsubsection{Analogue-to-Digital Converter}


\subsubsection{End-of-Travel Microswitches}


\subsection{Telemetry Module}


\subsubsection{Wireless Modem}

To meet the range and data throughput requirements for the telemetry
system, an XBee Pro wireles modem was used. The XBee requires 3.3v
I/O levels and power supply, and so a second linear voltage regulator
was used in the design, the LT1521 from Linear Technology. Since the
AT90CAN129 has only 2 built-in UARTS that were used for the RS232
interfaces to the ECU and DAQ, an third external UART was added to
the design. The MAX3100 is a SPI-interfaced UART with an 8 word deep
FIFO buffer. It is interfaced to the AT90CAN128's SPI pins and has
an active-low IRQ line connected to external interrupt line EXT7 on
the microcontroller. 


\subsubsection{External SPI USART}


\subsubsection{Two-Channel ECU and DAC USART}


\subsection{Driver Interface Module}


\subsubsection{Steering Wheel Unit}


\subsubsection{LCD Module Bias Circuit}


\subsubsection{LCD Module Data Interface}


\subsubsection{Input Knobs and Buttons}


\subsubsection{Paddle Shifters}


\section{Software Implementation\label{sec:Software-Implementation}}


\subsection{Methodology}


\subsection{Toolchain}


\subsection{Common Low-Level Drivers}


\subsubsection{CAN Driver}


\subsubsection{EEPROM Driver}


\subsubsection{SPI Driver}


\subsection{Engine and Transmission Module}

<Software interface map>


\subsubsection{Transmission Manager}


\subsubsection{Intake Manager}


\subsubsection{Starter Manager}


\subsubsection{Event Scheduler}


\subsubsection{CAN Interface}

<Data flow diagram>


\subsubsection{PWM Generator}


\subsubsection{Main Control Loop}


\subsection{Braking Module}


\subsubsection{Bias Library}


\paragraph{Bias Position Request}

<Position request flow chart>


\paragraph{Bias Calibration}

<Calibration flow chart>


\subsubsection{Pressure Library}


\paragraph{Periodic Pressure Output}


\paragraph{Pressure Calibration}

<Calibration flow chart>


\subsubsection{CAN Interface}

<Data flow diagram>


\subsubsection{Main Control Loop}


\subsection{Telemetry Module}

<Software interface map>

The primary objective of the software running on the Telemetry Module
is to push data around from various sources to various sinks. It uses
the interrupt-driven USART drivers written for the on-board USARTs
of the AT90CAN128, as well as the MAX3100 external USART extensively.
These USART drivers are discussed in section ?? 


\subsubsection{Internal USART Driver}


\subsubsection{MAX3100 Driver}


\subsubsection{Xbee Library}


\subsubsection{DAC Library}

To reduce the implementation work required for us, we asked David
Schilling, a computer science student on the Formula SAE team, to
write a software library to read and write the DAC's serial format,
given the requirements described in Section \ref{sec:Telemetry-Module-Design}.


\subsubsection{Main Control Loop}


\subsection{Driver Interface Module}


\subsubsection{LCD Module Library}


\subsubsection{Graphics and Fonts Library}


\subsubsection{I/O Library}


\subsubsection{Vehicle Diagnostics Library}


\subsubsection{Vehicle Parameter Library}


\subsubsection{Vehicle Dynamic Mode}


\subsubsection{CAN Interface}


\subsubsection{Main Control Loop}


\section{CAN Snooper and Injector}


\subsection{Overview}

<Picture of board>


\subsection{Scheduling}


\subsection{Injecting}


\subsection{Snooping}


\chapter{Operational Results}


\section{Transmission}


\subsection{Shift Speed}


\subsection{Launch Control}


\subsection{Neutral Find}


\subsection{Anti-Stall}


\subsection{Electropneumatics}


\section{Intake}


\section{Starter}


\section{Braking}


\subsection{Bias Adjustment}


\subsection{Calibration}


\section{Telemetry}


\subsection{Data Interfaces}


\subsubsection{ECU }


\subsubsection{DAC }


\subsubsection{Wireless Data Link}


\section{Driver Interface}


\subsection{Driver Controls}


\subsection{Vehicle Dynamic Adjustment}


\subsection{Diagnostic Information}


\subsection{Visually Appealing Interface}


\section{Electrical System and Harness}




\section{Implementation Issues Encountered}


\subsection{Hardware Implementation Issues}


\subsubsection{CAN Transciever Schematic Error}


\subsubsection{Driver Interface LCD Reset Line}


\subsubsection{Telemetry RS-232 Transciever Schematic Error}


\subsection{Interrupt Starvation on the Telemetry Module}


\subsection{CAN Driver Problems}


\chapter{Future Work}




\chapter{Conclusion}

\appendix

\chapter{Description of Attached Materials}

Attached DVD, etc.


\chapter{CAN Protocol Specification}


\chapter{Hardware Schematics}


\chapter{S-Series CAN Stream Specification}
\end{document}
