\section{Driver Interface Module\label{sec:Driver-Interface-Module}}
\nomenclature{LCD}{Liquid Crystal Display}

In previous years, the team has often been hindered by a lack of direct information from the electronic systems in the car. The driver interface module remedies this by providing information to the driver in real-time, and allows the driver to control aspects of the electronic control systems. The driver module and it's interactions are shown in Fig. \ref{fig:design_interface_overview_block}.

\begin{figure}[H]
	\centering
	\begin{tikzpicture}[auto, node distance=1.25cm, draw=black!70, >=stealth', font=\scriptsize]
  \node [block, blue shiny, text width=5cm] (module) {Driver Interface Module};
  \node [block, above of=module, text width=1cm, right=-2.5cm] (knobs) {Knobs};
  \node [block, above of=module, text width=1.25cm] (buttons) {Buttons};
  \node [block, above of=module, text width=1cm, left=-2.5cm] (paddles) {Paddles};

  \node [block, below of=module, text width=3cm] (panel) {Information Display Panel};

  \draw [->, thick] (knobs.south) -- ($(module.north west)!(knobs.south)!(module.north east)$);
  \draw [->, thick] (buttons.south) -- ($(module.north west)!(buttons.south)!(module.north east)$);
  \draw [->, thick] (paddles.south) -- ($(module.north west)!(paddles.south)!(module.north east)$);

  \draw [->, thick] (module) -- (panel);

  \draw [<->, thick] (module.east) -- ++(1cm,0) node[coordinate,label={[rotate=90]below:CAN Bus}](can){};
  \draw [-, line width=3pt] ($(can)+(0,-1cm)$) -- ++(0,2cm);

  \draw ($(knobs.north west)+(-0.2cm,0.2cm)$) node[coordinate] (x1) {};
  \draw ($(paddles.south east)+(0.2cm,-0.2cm)$) node[coordinate] (x2) {};
  \path [draw, thick, dashed] (x1) rectangle (x2);
  \node at (x1) [anchor=south west] {Driver Controls};
\end{tikzpicture}

	\caption{Overview of the driver interface module and it's interactions.}
	\label{fig:design_interface_overview_block}
\end{figure}

The driver module receives driver input from \emph{driver controls}, which are tactile inputs for adjusting vehicle dynamics. Driver commands are collected through the controls and relayed across the network to the relevant modules. The module communicates with the driver using the \emph{information display panel}, which displays vehicle information on an easy-to-read display mounted in the steering wheel. Diagnostic information from the other modules is received over the network and relayed to the driver through the information display panel. 

\subsection{Vehicle Dynamics Mode (VDM)}
\nomenclature{VDM}{Vehicle Dynamics Mode}

The driver interface system provides a method of quickly modifying several dynamic vehicle parameters quickly and easily. We have called this feature \emph{Vehicle Dynamics Mode} (VDM). 

For example, an acceleration event calls for launch control, auto-upshift, and a heavy forward brake bias. It is possible to enable all of these features in one step by changing the VDM mode to {}``acceleration''. 

When a new mode is selected, all nodes on the network are notified and synchronized to modify their dynamic parameters in accordance with the specific mode. For example, when the  {}``acceleration'' mode is enabled, the engine module will enable launch control and auto-upshift, and the brake controller will modify the brake bias to a pre-set ratio.

\begin{description}
  \item [{Pit Mode}] enables soft-launch driving characteristics that mimic a fully automatic transmission. This makes slowly driving the car forward from a stand-still far easier, and only requires the driver to take their left foot off the brake, and slightly apply the throttle.
  \item [{Acceleration Mode}] puts the vehicle systems into full-performance characteristics. Launch control is activated. The engine module will watch for a launch signal from the driver, and will automatically up-shift based on the engine RPM.
  \item [{Dynamic Mode}] puts the vehicle systems into a mode that is suitable for the autocross, and the endurance race.
\end{description}

Individual VDM parameters can be changed by the driver using the driver controls. Existing presets may be overwritten on demand with the new parameters chosen by the driver.

\subsection{Driver Controls}
\label{sec:interface_controls}

The driver controls consist of adjustment knobs, buttons, and paddle shifters. See table \ref{table:driver_controls} for a description of each.

\begin{table}[H]
\caption{Available driver controls.}
\centering
\begin{tabular}{|c|c|p{8 cm}|}
	\hline 
	Type & Name & Description \\
	\hline
	\hline 
	Knob & VDM & Changes the current VDM setting. \\
	\hline 
	Knob & Option & Changes the currently selected menu option. \\
	\hline
	Knob & Adjust & Adjusts the value associated with the currently selected menu option. \\
	\hline
	Button & Starter & Engages the automatic start sequence if pressed once, or the manual start sequence if held down.\\
	\hline
	Button & Neutral Find & Activates the neutral-find feature. \\
	\hline
	Button & Diagnostics/Save & Pages through diagnostic information if pressed once, or saves the current dynamic settings to the selected VDM if held down.\\		
	\hline 		
	Paddle & Up-Shift & Up-shifts the transmission. \\
	\hline
	Paddle & Down-Shift & When held, disengages the clutch and down-shifts. When released, re-engages the clutch. \\
	\hline
\end{tabular}
\label{table:driver_controls}
\end{table}

\subsection{Diagnostic Information}
\label{sec:interface_diag}

The interface display panel is a large monochrome \emph{Liquid Crystal Display} (LCD) screen, inset in the steering wheel. The panel provides real-time information to the driver regarding all of the electronic systems in the car. Primary information displayed on the panel includes:

\begin{itemize}
\item the selected gear;
\item the active \emph{vehicle dynamics mode} (VDM);
\item the telemetry signal strength;
\item the engine RPM;
\item the vehicle wheel speed; and
\item the status of the launch control feature.
\end{itemize}

The list of available options and the current value for the selected option is displayed on screen when the driver rotates the option dial. After a timeout period, normal real-time information display resumes. 

\subsection{Hardware}

A high-level overview of the module's hardware design is shown in figure \ref{fig:interface_hardware_design_block}. The driver interface module builds on the common architecture present in the other three modules. 

\begin{figure}[H]
\centering
\begin{tikzpicture}[auto, node distance=2cm, draw=black!70, >=stealth']
  \node[block, text width=1.25cm, minimum height=1cm] (lcd) {LCD\\ Module};
  \node[block, right of=lcd, minimum height=1cm, text width=1.15cm] (knobs) {Knobs};
  \node[block, right of=knobs, minimum height=1cm, text width=1.35cm] (buttons) {Buttons};
  \node[block, right of=buttons, text width=1.35cm, minimum height=1cm] (paddles) {Shift\\ Paddles};

  %%% Microcontroller block
  \path ($(lcd.south west) + (0,-1.25cm)$) node[coordinate] (mu1) {};
  \path ($(paddles.south east) + (0,-1.25cm)$) node[coordinate] (mu2) {};

  \node [block, fit=(mu1) (mu2), inner xsep=0, minimum height=1cm] (micro) {Microcontroller};
  \node at ($(micro.east)+(2cm,0)$) [block, text width=2cm] (can_interface) {CAN Transceiver};

  \draw [<->, thick] (lcd.south) -- ($(micro.north west)!(lcd.south)!(micro.north east)$);
  \draw [<->, thick] (knobs.south) -- ($(micro.north west)!(knobs.south)!(micro.north east)$);
  \draw [<->, thick] (buttons.south) -- ($(micro.north west)!(buttons.south)!(micro.north east)$);
  \draw [<-> ,thick] (paddles.south) -- ($(micro.north west)!(paddles.south)!(micro.north east)$);

  \draw [-, line width=3pt] (can_interface.north) -- ++(0, 1cm) node[coordinate,label=above:CAN Bus](x1){} -- ++(1cm,0);
  \draw [-, line width=3pt] (x1) -- ++(-1cm,0);
  

  %%% Background
  \begin{pgfonlayer}{background}
    \path (micro.north west)+(-0.3,0.3) node (a) {};
    \path (can_interface.south east)+(+0.2,-0.2) node (b) {};
    \path[module] (a) rectangle (b);
  \end{pgfonlayer}
\end{tikzpicture}

\caption{A block diagram of interface module hardware.}
\label{fig:interface_hardware_design_block}
\end{figure}

\subsubsection{LCD Module}

The LCD module is a self-contained unit that consists of an easy-to-read LCD screen and a controller for interfacing the LCD with a micro-controller. It is not directly attached to the driver interface module, to help decouple the driver interface module and steering wheel design.

\subsubsection{Knobs, Buttons, and Paddles}

The knobs and buttons described in Sec. \ref{sec:interface_controls} connect to the driver interface module. Appropriate de-bouncing and knob encoding circuitry is present to interface the controls with the module. 

\subsection{Software}

A block-diagram overview of the software design is shown in Fig. \ref{fig:interface_software_design_block}. Like with other designs, various managers oversee operation of the module, and interact with hardware through abstraction interfaces.

\begin{figure}[H]
	\centering
	\tikzstyle{big arrow} = [>=latex, line width=4pt, gray]

\begin{tikzpicture}[auto, node distance=1.75cm, draw=black!70, >=stealth', font=\scriptsize]
  \node [block, grey shiny, minimum width=4cm, inner xsep=0, text width=4cm] (vdm_manager) {VDM Manager};
  \node [block, grey shiny, minimum width=4cm, inner xsep=0, text width=4cm, below of=vdm_manager] (controls_manager) {Controls Manager};
  \node [block, grey shiny, minimum width=4cm, inner xsep=0, text width=4cm, below of=controls_manager] (ui_manager) {User Interface Manager};
  \node [block, grey shiny, minimum width=4cm, inner xsep=0, text width=4cm, below of=ui_manager] (diag_manager) {Diagnostic Manager};

  \node [block, left of=vdm_manager, text width=1.5cm, node distance=4cm] (nv_storage) {NV Storage Interface};
  \node [block, left of=controls_manager, text width=1.5cm, node distance=4cm] (gpio) {GPIO Interface};
  \node [block, left of=ui_manager, text width=1.5cm, node distance=4cm] (lcd) {LCD Interface};

  \node [block, right of=controls_manager, node distance=4cm, text width=1.5cm] (can_interface) {CAN Interface};

  \draw [<->, big arrow] (vdm_manager) -- (nv_storage);
  \draw [<->, big arrow] (controls_manager) -- (gpio);
  \draw [<->, big arrow] (ui_manager) -- (lcd);

  \draw [<->, big arrow] (controls_manager) -- (ui_manager);
  \draw [<->, big arrow] (ui_manager) -- (diag_manager);

  \draw [<->, big arrow] (controls_manager) -- (can_interface);
  \draw [<->, big arrow] (can_interface) |- (vdm_manager);
  \draw [<->, big arrow] (can_interface) |- (diag_manager);
\end{tikzpicture}

	\caption{The interface module software block diagram.}
	\label{fig:interface_software_design_block}
\end{figure}

\subsubsection{User Interface Manager}

The \emph{user interface manager} controls the content displayed on the information display panel, and implements a menu system for the driver to modify the various dynamic parameters of the vehicle. New input from the driver is directed to the interface for processing. Vehicle diagnostics are also relayed to the manager and displayed as appropriate. The user interface manager updates the information display panel through the \emph{LCD interface}.

\subsubsection{Diagnostics Manager}

The \emph{diagnostics manager} listens to the network for the information listed in Sec. \ref{sec:interface_diag}. It grabs important vehicle parameters and coordinates with the user interface manager to display them on the information display panel.

\subsubsection{Vehicle Dynamic Mode (VDM) Manager}

The \emph{vehicle dynamic mode manager} handles requests from the driver control manager to change the current vehicle dynamic mode, and to overwrite the dynamic mode presets. It uses the CAN interface to communicate the new parameters over the network to the other modules. The non-volatile storage interface is utilized to load and save the parameters for each preset. 

\subsubsection{Driver Controls Manager}

The \emph{driver controls manager} listens for new driver inputs on the knobs, buttons, and paddles. Driver input is directed to the user interface manager for processing. Transmission requests are relayed directly to the engine and transmission module over the CAN interface.

\subsubsection{LCD Interface}

The \emph{LCD interface} abstracts the functionality of the LCD module for the user interface manager. It translates high-level drawing and text commands into the low-level memory operations required by the LCD module to draw on the screen.

\subsubsection{General Purpose Input-Output (GPIO) Interface}

The \emph{general-purpose input-output (GPIO) interface} monitors the knobs, buttons, and levers for activity. It notifies the driver control manager when a change of state occurs. 

\subsubsection{Non-Volatile Storage Interface}

The \emph{non-volatile storage interface} reads and writes vehicle dynamic mode parameters to the non-volatile storage portion of the micro-controller. 


