\section{Braking}

The most desired improvement for the braking system is to eliminate the need for adjusting the brake bias manually, which requires substantial effort and time on behalf of the pit crew. Thus, the main goal for the braking system is to introduce an electronically-adjustable bias system. As the braking system is subject to mechanical wear and inconsistency, a secondary goal is to provide the ability to recalibrate the braking system at will.

\subsection{Bias Adjustment}

Eliminating the need to manually adjust the bias will improve the tuneability of the vehicle. The practical range of adjustment is a force distribution ratio of 70\% to 30\%, either front-back or back-front. For safety reasons, the driver may not adjust the brake bias while the vehicle is in motion. Adjusting the bias also requires that there be no pressure in the brake lines.

\begin{itemize}

\item Allow the driver to adjust the brake bias from a control in the cockpit.
\item Provide a front-rear adjustment range of 30/70 to 70/30.
\item The bias must not be adjustable while the vehicle is in motion or while the brakes are applied.

\end{itemize}

\subsection{Bias Calibration}

Because the system is subject to wear and inconsistency, the crew should be able to re-centre the balance bar as needed. This should be a short procedure, as there is usually not much time between events at a competition.

\begin{itemize}

\item Provide the ability to recalibrate the bias adjustment system to account for mechanical wear and inconsistency. 
\item It must take less than 30 seconds to complete the bias adjustment calibration.

\end{itemize}
